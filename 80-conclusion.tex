%$tex

\Conclusion % заключение к отчёту

Чин пострижения черниц ангельского образа -- яркий, самобытный и особенно интересный как сохранивший и древние особенности монашеских чинов, и, вероятно, ещё более древнюю литургическую образность обручения с Женихом.
Он вызывает и исследовательский интерес как совмещающий в себе множество элементов различных богослужений: чин омовения ног, чин малой схимы, чин великой схимы, антифонное шествие кафедрального последования литургии.
Он интересен и сам по себе как ярко образный, динамичный, особенно благодаря двум антифонным шествиям и входам-выходам.

Чину присуща определённая поэзия, вероятно не только сохранённая, но и доосмысленная и обогащённая особенностями как русской, так и женской души.
Образы, знаки, символы в чине поэтически переплетаются: свеча в руке и Богородица как свеча, образы совлечения ветхой одежды и движение в одной сорочке к облачению в алтаре, чертог Жениха искомый возлюбленной и вход в алтарь, где на престоле -- Св. Дары.

С другой стороны, уже в своём первоначальном виде чин являет проблемы, характерные для смешения, расширения в «слабых местах»\cite{taft.rastut.2009} и последующего урезания важных смысловых моментов.
Так, главная бросающаяся в глаза проблема -- смысловая разделённость, параллельность совершающегося пострига и литургии, при их внешнем динамическом объединении.
Другая проблема, которую чин унаследовал от предшественников -- запутанность соединения чинов малой и великой схимы.
Спор о необходимости слияния ступеней монашеского пострига в одну держится с эпохи преп. Феодора Студита и до наших дней, а попытки то разделить, то объединить чины, не пошли им на пользу.


%%% Local Variables: 
%%% mode: latex
%%% TeX-master: "rpz"
%%% End: 

