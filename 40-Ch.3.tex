\chapter{Реконструкция истоков чина}

\section*{Крещение}

В «оглашении» в чине великой схимы есть слова «О новаго звания! О дара тайны! Второе крещение приемлеши днесь, брате, богатствомъ человеколюбца Бога даровъ, и отъ греховъ твоихъ очищаешися, и сынъ света бываеши», а в молитве «Сый Владыко Вседержителю, вышний Царю славы» -- «Сопричти его избраннымъ Твоимъ, да будетъ Твой сосудъ избранный, сынъ и наследникъ Царствия Твоего\ldots{}».
Эти и другие черты сближают чин пострига в великую схиму с крещением.

Согласно Н. Пальмову\cite{palmov.postrijeniye.1914}, эти элементы появились в современном типе чина благодаря редакции преп. Феодора Студита.
В X--XIвв. чин уже имел такой вид, согласно Евхологию Московского Румянцевского Музея № 474 и Евхологию 1027г. Парижской Национальной (Coislin) библиотеки.
В более ранней редакции, в Евхологии Барберини св. Марка эти элементы отсутствуют и чин имеет другой облик.

Впрочем, эти элементы присутствуют и в мужском постриге.
Разница в тематизме песнопений: в женском чине образ сочетания со Христом выступает гораздо явственнее.

\section*{Чины хиротоний}\label{ux447ux438ux43dux44b-ux445ux438ux440ux43eux442ux43eux43dux438ux439}

Облачение в алтаре более всего напоминает чины поставления на служение.

Введение в алтарь встречается уже в наиболее раннем из известных нам чинов пострижения в Евхологии Барберини\cite[С.~210–216]{barberini.evhologiy.2011}.
Этот чин, один из древнейших известных чинов монашеского пострига, состоит из двух частей: малого и великого пострига.
В ходе малого пострига, «отрекающийся входит в святилище и приклоняется перед святым жертвенником», что бывает также в чинах рукоположения, в т.ч. диаконисс\footnote{Правда, диакониссы при рукоположении лишь приклоняли главу. Диакон вставал на одно колено, а пресвитер -- на оба.}.
О рукоположении диаконисс в алтаре пишет Неселовский\cite[С.~72]{neselovskiy.4ini.1906}, ссылаясь на Матфея Властаря, основного свидетеля по этому вопросу.
Далее, одна из молитв древнего чина имеет слова: «Господи Боже истины, именем твоим \textbf{возлагаю руку мою на твоего раба} \emph{имярек}\ldots{} чтобы \textbf{служил всегда твоей благости}, поклоняясь и восхваляя твоё имя святое, прославляя тебя во все дни его жизни\ldots{}».

В чине пострижения черниц ангельского образа осталось простирание перед царскими вратами (и алтарём) перед постригом, после чего священник поднимал постригащуюся за руку.
Также, присутствует облачение в алтаре.
Антифоны сохранили прошения о духовных дарованиях.

Облачение завершается перед анафорой, и в некоторых источниках сообщается, что новому схимонах(у/ине) давалась свеча, с которой он(а) участвовал(а) в великом входе со свечой\cite{palmov.postrijeniye.1914} (вероятно, предстоя перед вратами как делают сейчас алтарники).

Хотя маловероятно, что непосредственно хиротония диаконисс повлиял на женский чин великой схимы, но в целом родство с чинами поставления на служение прослеживается.
Присутствует даже своеобразное литургическое «служение» (свеча).

\section*{Чин поставления дев}\label{ux447ux438ux43d-ux43fux43eux441ux442ux430ux432ux43bux435ux43dux438ux44f-ux434ux435ux432}

Красносельцев упоминает возможные истоки образности песнопений чина и более ранних молитв из Евхология Барберини\footnote{Имеются ввиду шесть молитв №258--263 «на пострижение монахинь»\cite[С.~176–178]{barberini.evhologiy.2011}} в чинах поставления дев и вдовиц церковных\cite[С.~145–146]{krasnoseltcev.uprazdnenniye.1889}.
В обоих случаях основным является образ обручения с небесным женихом и ожидания его с горящим светильником.

Девы и вдовицы не рукополагались, о чём сказано в Апостольском предании, среди упоминания других церковных чинов (Trad. Ap. 12), однако в IVв. на Западе формируется velatio\footnote{В дальнейшем чин дев упоминается в произведениях св. отцов на Востоке до Vв.. Вероятно, в дальнейшем он был вытеснен женским монашеством и диакониссами, разделившими его функции. На Западе сохранялся особый чин посвящения в дев до XIIIв., который был возрождён в 1970г. в результате II Ватиканского Собора.} -- особый чин посвящения дев через возложение плата им на голову и чтение молитв\cite{pravenc.devy.20}.
В 6-м правиле Карфагенского поместного собора предписывается «освящение дев» не творить пресвитеру и многие комментаторы, в т.ч. Красносельцев трактуют это как возможность совершать чин только епископу.

Согласно Красносельцеву, обряд посвящения в девы состоял из принесения обета девства, «увещания епископа», возложения на голову посвящаемой покрывала.
Это покрывало -- flammeum virginale.
Согласно античной традиции, при вступлении в брак, на голову женщины надевался плат, называемый flammeum (или velum).
Эта деталь обряда бракосочетания вошла в обряд посвящения в девы, став в нём вещественной символикой брака.\\
Девы назывались «невестами Христовыми» начиная с Тертуллиана.

Как минимум до запрета 45-м правилом Шестого вселенского собора, существовал обычай перед пострижением одевать женщину в дорогие костюмы и украшения, словно невесту, и в таком виде приводить в храм.

В чине поставления дев, представленном в сакраментарии Геласия (ок. 750г.), присутствуют молитвы, которые с незначительными изменениями вошли в состав современного посвящения в девы, восстановленного католической церковью после II Ватиканского собора.
В этих молитвах также присутствует мотив Жениха-Христа: «They give themselves wholly to Christ, the Son of the ever-virgin Mary, and the heavenly Bridegroom of those who in hist honor dedicate themselves to lasting virginity\footnote{Здесь и далее молитвы из чина поставления дев цитируются по новому католическому чину, на английском языке.}» и даётся образ рождения от Духа: «Your children are born, not of human birth, nor of man's desire, but of your Spirit».
Но, значительно больше внимания уделяется образу девства: «They place in your hands their resolve to live in chastity\ldots{} Among your many gifts you give to some the grace of virginity\ldots{} Those who choose chastity have looked upon the face of Christ, its origin and inspiration\ldots{} in his honor dedicate themselves to lasting virginity\ldots{}» -- четыре и даже более упоминания девственности втечение одной молитвы!
И Христос также описан как образец и вдохновение для целибата.

Так, в чине поставления дев основное внимание оказывается на личном подвиге девства, а не на браке со Христом, как в постриге черниц ангельского образа.
В обоих чинах присутствуют прошения на укрепление и защиту от искушений, но в чине дев отсутствует образ брани духовной и ожидания со свечой в руках Жениха.
Последний образ может быть особенно важной отличительной чертой русского чина, потому как сообщает ему эсхатологическое измерение и дополняет яркую тему влюблённости духовной трезвостью взыскующей нетленного царства.

Помимо этих отличий, есть и важное сходство, касающееся вне-литургического аспекта обоих чинов.\\
Одна из задач чина дев, согласно свт. Амвросию Медиоланскому, -- укрепление для преодоления сопротивления христианству в семье: «та, которая победила дом, победила век» (Ambros. Mediol. De virginib. 1. 12. 64).
А согласно 140 правилу Карфагенского Собора, -- и для укрепления «в опасности угрожающей целомудрию девы, когда есть подозрение или о любителе сильном, или о каком либо похитителе».

Итак, выделяются две тематические линии: 1) бракосочетание с Женихом-Христом и 2) укрепление уз семьи духовной для преодоления сопротивления как собственной плоти, так и семьи по плоти.
Вероятно, обе они оказывались актуальными и на Руси в среде женского иночества XIV--XVIIвв..

XIX век как время особого раскола между церковным и светским обществом, вдвойне привлекал внимание к этим темам.
Первая привлекала своим мистицизмом уставшую от просвещенческого рационализма интеллигенцию, вторая оказывалась актуальна для верующих женщин в среде светского небрежного, а иногда и агрессивного отношения к церкви.
Здесь можно вспомнить такие произведения как роман «Дворянское гнездо» Тургенева, где героиня против воли семьи уходит в монастырь, картину Нестерова «На горах» и многие другие его картины, связанные с образами монашества и великой схимы.
Наконец, в романе Мельникова-Печерского «На горах» даётся довольно выпуклый образ на эту же тему: читатель видит чин глазами купца, страстно влюблённого в монахиню и надеющегося похитить её из монастыря.
Когда он признаёт в постригающейся ту самую, кого чаял похитить, его мечты рассыпаются.

