%$tex

\Introduction

\textbf{Цель} данной работы -- литургико-исторический анализ особого чина пострижения в великую схиму женщин, известного также как \emph{пострижение черниц ангельского образа}, бытовавшего на Руси в XIV--XVII вв..

Для достижения поставленной цели необходимо:

\begin{itemize}
\tightlist
\item
  описать структуру и состав чина по его представлению у Н. Красносельцева и по оригинальным источникам
\item
  проанализировать особенности, отличающие чин от мужского пострига в великую схиму
\item
  проанализировать возможные предпосылки к формированию чина в чинопоследованиях крещения, хиротоний и поставления дев
\end{itemize}

\textbf{Актуальность} выбранной темы связана с тем, что последнее подробное описание чина сделано Н. Ф. Красносельцевым в 1889г., с тех пор стало известно больше источников, в т.ч. описан Дмитриевским Евхологий 1027г. №213, исследован и восстановлен в католической церкви чин поставления дев и т.д..
Также, настоящее время предъявляет всё больше вопросов как к христианскому пониманию иночества, так и к женскому служению в церкви.
Чин пострижения черниц ангельского образа позволяет исследовать как определённые, сохранившиеся в древнем чине антропологические воззрения на иночество и служение, так и их гендерное преломление.

\textbf{Степень изученности и источники}.
В русских требниках от XV и до сер. XVII вв. совместно с чином пострижения в великую схиму мужчин, можно найти особый чин для постригающихся женщин, «чин пострижения черниц ангельского образа».
Он встречается в рукописях, хранящихся в Соловецкой библиотеке, библиотеке Троице-Сергиевой Лавры и в более поздних печатных Московский потребниках: Филаретовском (1625г.), Иноческом (1639г.), Иосифофском (1651г.).
В ходе книжной справы 1640-60-х гг., чин исчезает, т.к. отсутствует в греческих книгах, по которым проводилась справа.
В настоящее время для мужского и женского пострига предназначен один чин.

Женский чин пострижения в великую схиму был найден вновь и описан Н.В. Красносельцевым как «чин пострижения черниц»\cite[С.~134–169]{@krasnoseltcev.uprazdnenniye.1889} в ходе исследования рукописей Соловецкой библиотеки.
Несмотря на исчезновение из требников, чин как и монашеская традиция вызывали большой интерес в XIX веке, о чём может свидетельствовать его описание в реалистическом романе «На горах» П. Мельникова-Печерского\footnote{Роман «На горах» написан в 1874-75 гг., что ненамного, но опережает выход книги Н. Красносельцева с описанием чина -- в 1889 г.. Вероятно, он провёл самостоятельные исследования, а также мог получить консультацию от бывшего старообрядца, уставщика Рогожского кладбища Василий Борисов, ставший прототипом Василия Борисыча в его романах. \textit{За этот факт автор работы благодарит Боченкова В., автора книги о Мельникове–Печерском.}}, с подробным и правильным приведением начальных строк особых стихир и антифонов чина, а также верно описанной структурой.

Наиболее раннее изложение этого чина было найдено проф. А. Дмитриевским в Евхологии 1027г. Парижской Национальной (Coislin) библиотеки №213\cite[С.~1035–1042]{dmitriyevskiy.opisaniye.1901}.
В Русской Церкви наиболее ранний источник -- славяно-русский Требник XIVв. Московской Троице-Сергиевой лавры №229, ЛЛ. 29--35, где он появляется под названием «Служба скымихя на пострижениiе черницам».
Согласно гипотезе Н. Пальмова, он мог попасть в русскую церковь благодаря юго-славянскому влиянию на русское богослужение XIVв.\cite[С.~14]{palmov.postrijeniye.1914}, чему дополнительное свидетельство -- аналогичный чин с похожим названием «Служба на пострижениiе черноризици» в юго-славянском Требнике Имп. СПБ. Публ. библ. №21 из собрания А. Ф. Гильфердинга, ЛЛ. 37--51.


