\chapter{Чин, его структура и составные элементы}\label{ux433ux43bux430ux432ux430-1-ux447ux438ux43d-ux435ux433ux43e-ux441ux442ux440ux443ux43aux442ux443ux440ux430-ux438-ux441ux43eux441ux442ux430ux432ux43dux44bux435-ux44dux43bux435ux43cux435ux43dux442ux44b}

\section*{Общие замечания}\label{ux43eux431ux449ux438ux435-ux437ux430ux43cux435ux447ux430ux43dux438ux44f}

Чин пострижения черниц ангельского образа во многом повторяет мужской постриг в великую схиму, но с некоторыми заметными изменениями.
Во-первых, служба предваряется обрядом омовения ног постригающейся, который совершает игуменья.
Во-вторых, многие песнопения изменены по сравнению с мужским постригом.

В данной работе, как и в источниках, слова «как в мужском чине пострига» отсылают к постригу мужчин в великую схиму в ту же эпоху (XIV--XVIIвв.).
Существующий в настоящее время единый чин великой схимы отличается.

\section*{Омовение ног}\label{ux43eux43cux43eux432ux435ux43dux438ux435-ux43dux43eux433}

Начинается обряд после утрени, когда в келью желающей принять постриг приходят инокини.
Песнопения в этом месте специально подобраны к женскому чину.
Что интересно, они непосредственно обращены к постригающейся.
После первой стихиры, в чине появляется священник (здесь непонятно, вышли ли из кельи к нему, или вошёл он вместе с сёстрами).

Структура последования (выделены действия):

\begin{itemize}
\tightlist
\item
  \textbf{инокини приходят в келью к постригаемой}
\item
  инокини поют стихиру: «Послушай Христа, что вопиет дево, крест вземше иго мое на раме»
\item
  священник творит обычное начало
\item
  Псалом 50
\item
  Канон великой схимы
\item
  \textbf{игуменья за руку ведёт постригаемую} в притвор церкви, \textbf{монахини со свечами следуют за ними и поют} стихиру: «Последуимъ сестры благому владыце, увядимъ мiрскихъ похотей» и богородичен: «Грешныхъ молитвы прiемлюще скорбящихъ въздыханiе»
\item
  поют три антифона, третий («Союзомъ любви связуеми апостоли») взят из чина омовения ног в Великий Четверг
\item
  ектенья
\item
  две молитвы из чина омовения ног
\item
  Евангелие от Иоанна, в ходе чтения \textbf{игуменья препоясывается «понявою» и умывает ноги желающей принять пострижение и отирает их «лентiемъ»}
\item
  во время омовения ног, трижды поют стихиру: «Умый нозе честная мати и обуй мя сапогомъ целомудрiя, да не пришедъ врагъ обрящетъ пяты моя нагы и запнетъ стопы моя» и богородичен: «Имуще тя, Богородице, упованiе»
\item
  дочитывается Евангелие (вероятно, имеется ввиду зачало от слов «Когда же Он умыл их ноги и взял одежду Свою и возлёг снова, Он сказал им: знаете ли, что Я сделал вам? (Ин 13:12))
\item
  поётся тропарь «Господа глашаете мя и учителя» со стихом «Окропиши мя иссопомъ», \textbf{игуменья и монахини кропят «на лица своя отъ воды умыванiя»}
\end{itemize}

\section*{Литургия слова и пострижение}\label{ux43bux438ux442ux443ux440ux433ux438ux44f-ux441ux43bux43eux432ux430-ux438-ux43fux43eux441ux442ux440ux438ux436ux435ux43dux438ux435}

После омовения ног в притворе, игменья, постригаемая и сёстры входят в церковь.
По входу в церковь, начинается литургия с чином пострижения в великую схиму и особыми антифонами, тематически подобранными к женскому постригу.
По ходу пения антифонов, постригаемая движется к алтарю, с тремя остановками: в начале (притворе), у амвона и перед царскими вратами.

Постриг продолжается в два этапа: сперва постригает священник, а затем сёстры -- в дьяконнике или на паперти.
Так же происходило и в мужском чине.
Иногда между постригами было целование постригае(мой/ого).
Иногда второй этап пострига совершался при участии небольшого числа братьев (сестёр), а именно восприемников.

\begin{itemize}
\tightlist
\item
  \textbf{игуменья, постригаемая и сёстры входят в церковь с пением} тропаря «Къ тебе утреннюемъ милосердому умывше нозе» и богородичен: «Избави насъ от бедъ нашихъ мати Христа Бога»
\item
  великая ектенья
\item
  1-й антифон: «Хотех слезами омыти моихъ греховъ, Господи, рукописанiе\ldots{}»
\item
  \textbf{постригаемая подходит к амвону}
\item
  2-й антифон: «Где есть мирская краота, где есть временныхъ мечтанiе\ldots{}»
\item
  \textbf{постригаемая подходит к царским вратам}
\item
  3-й антифон: «Господи отверзи двери чертога рабе твоей\ldots{}»
\item
  «Придите, поклонимся»
\item
  на «Придите, поклонимся», \textbf{постригаемая простирается ниц у царских врат} до конца богородична
\item
  стихира «Прiидите ко мне вси труждающiеся» и богородичен «Иже тебе ради Богоотецъ пророкъ Давидъ»
\item
  следуют \textbf{вопросы священника и ответы постригаемой}, те же, что и в мужском постриге, начиная с: «Что прiиде, сестро, припадая къ святому жертвеннику?»
\item
  \textbf{пострижение}
\item
  \textbf{сёстры вводят постригаемую в диаконник или на паперть и постригают}, воспевая целиком 17-ю кафизму («Блаженны непорочные»)
\item
  в это время читаются Апостол и Евангелие, что и в мужском постриге
\end{itemize}

\section*{Литургия верных и облачение}\label{ux43bux438ux442ux443ux440ux433ux438ux44f-ux432ux435ux440ux43dux44bux445-ux438-ux43eux431ux43bux430ux447ux435ux43dux438ux435}

После пострига, сёстры вновь ведут новопостриженную в церковь.
Следует её второе приближение к алтарю с тремя новыми антифонами и остановками в тех же местах.
В этот раз по приближении к царским вратам, её вводят в алтарь и начинается облачение.

\begin{itemize}
\tightlist
\item
  \textbf{новопостриженную сестру вводят в церковь} «въ конечней ризе\footnote{Имеется ввиду срачица.}, неопоясану, непокровенну, необувенну»
\item
  3 антифона «Евангельского гласа вскоре повеленiе исполнила еси»
\item
  во время пения антифонов \textbf{новопостриженная приближается к алтарю и к третьему встаёт у царских врат}
\item
  \textbf{новопостриженная входит в алтарь}
\item
  \textbf{священник облачает её в великосхимнические одежды}, комментируя каждый предмет облачения. В это время сёстры поют стихиры, что и в мужском пострижении
\item
  молитвы священника: Господи Боже нашъ введи рабу твою» и «Господи Боже нашъ, верныхъ въ обетованiи»
\item
  стихиры «Да возрадуется душа моя о Господе», «Сорадйтеся ми о Христе сестры» и др.
\item
  \textbf{целование Евангелия и новопостриженной}: дьякон встаёт в царских вратах, держа Евангелие вне врат. Новопостриженная целует Евангелие и встаёт рядом с дьяконом. Игуменья и сёстры по очереди целуют Евангелие, а затем новопостриженную и трижды поют стихиру на 1-й глас: «Разумеимъ сестры таинства силу» и богородичен «Радуйся Богородице, дево, радуйся похвало»
\item
  литургия далее завершается как обычно
\end{itemize}

\section*{Трапеза и третье введение в церковь}\label{ux442ux440ux430ux43fux435ux437ux430-ux438-ux442ux440ux435ux442ux44cux435-ux432ux432ux435ux434ux435ux43dux438ux435-ux432-ux446ux435ux440ux43aux43eux432ux44c}

В завершение пострижения игуменья, новопостриженная и сёстры покидали храм для совершения трапезы.
После трапезы, игуменья вновь вводила за руку новопостриженную в церковь, где та оставалась 7 дней до молитвы на снятие кукуля.
На 8-й день над ней читается та же молитва на снятие кукуля, что и в мужском постриге.

\begin{itemize}
\tightlist
\item
  после отпуста литургии \textbf{сёстры с новопостриженной выходят из церкви на трапезу с пением} тех же стихир, что и в мужском постриге
\item
  по окончанию трапезы поются стихиры «Аще внешняго совлечеся одеянiя» и «Премудрый Павелъ невестоводитель бываетъ на небесехъ тайно принося мир обрученiе Христово»
\item
  \textbf{игуменья за руку вводит новопостриженную в церковь в третий раз}
\item
  в это время сёстры поют стихиры «Господи, Господи, призри съ небесе и виждь» и богородичен «Свеще неугасимая»
\item
  в церкви поются несколько песнопений, специально выбранных для женского пострижения, последнее из которых «Чистъ ест бракъ и нескверненъ, о дево, неугасиму свещу соблюдай всегда»
\item
  по окончанию пения \textbf{сёстры расходятся по кельям, а новопостриженная остаётся в церкви}
\end{itemize}
