\chapter{Особенности действий и песнопений в чине}\label{ux433ux43bux430ux432ux430-2-ux43eux441ux43eux431ux435ux43dux43dux43eux441ux442ux438-ux434ux435ux439ux441ux442ux432ux438ux439-ux438-ux43fux435ux441ux43dux43eux43fux435ux43dux438ux439-ux432-ux447ux438ux43dux435}

\section*{Омовение ног}\label{ux43eux43cux43eux432ux435ux43dux438ux435-ux43dux43eux433-1}

Обряд омовения ног является вступлением к постригу.
В мужских чинах ничего подобного не встречается.
По структуре и содержанию молитв, чтений, песнопений, действий, он почти полностью повторяет омовение ног в кафедральных соборах в Великий Четверг.
Обряд в этот день совершался также и в монастырях, где как и в исследуемом чине, его совершал игумен(ья)\cite[С.~149]{krasnoseltcev.uprazdnenniye.1889}.

По поводу соединения чина с постригом, Красносельцев предполагает причину в приурочении чинов поставления дев к определённым праздникам (Пасха, апостолов Петра и Павла, Богоявления), подобно крещениям.
Об этом пишет и св. Амвросий Медиоланский в Exhortatio virginatis: «пришёл день Пасхи: во всём мире совершаются таинства крещения, возлагаются покрывала на священных дев».
Великий четверг же оказывался днём, к которому приурочивали множество священнодействий, в т.ч. совершаемых епископом: освящение мира, воссоединение кающихся, общее елеосвящение.
Вероятно, и освящение дев, которое тоже мог творить только епископ, совершалось в этот день.
Так, соединённые в записях, чины могли слиться, когда были забыты изначальные смыслы их нахождения рядом\cite[С.~150]{krasnoseltcev.uprazdnenniye.1889}.

Омовению ног предшествуют не встречающееся в чине Великого Четверга песнопения.
По входу в келью желающей принять постриг, сёстры поют стихиру:

\textit{Послушай Христа, что вопиетъ, о дево. крестъ вземши, иго мое на раму, иди отвержися земныхъ. да не привлечетъ тебе страсть. ни пища чревная тя лишитъ, истиннаго живота. внемли, утвержай страшное обещание. где пасетъ, где пребываетъ. егоже любитъ душа моя.
}

Потом, с зажжёнными свечами, следуя за игуменьей и постригаемой в притвор:

\textit{Последуемъ сестры благому владыце, увядимъ мирския похоти. бежимъ лестьца мира, и миродержителя. будемъ чисти и совершени, почтемъ образъ, устыдимся звания. преложимъ житие. что творимся сами смирени, высоцы бывше. что в видимыхъ пребываемъ. темъ каяждо насъ себе плодъ принесетъ Христови, по данней ей благодати. давсякими мерами добродетелей, тамо сущая славы обители наследуемъ.}

Небольшие изменения касаются объекта прошений на ектенье и в молитвах, который из «нас» превращается в «её».
Так, в молитве «Боже преблагий, неприступный божествоъ, иже во зраце рабий служителя образъ восприемый», появляется «греховныя скверны омы ти рабы своея, \emph{имрк.}, сподоби прикосновениемъ воды сея. да чиста душею же и теломъ бывши, благоугодно ти послужит\ldots{}».

\section*{Антифоны}\label{ux430ux43dux442ux438ux444ux43eux43dux44b}

Как уже было сказано, песнопения в чине значительно отличаются по тематики от используемых в мужском постриге на Руси в это время.
Последние объединены темой покаяния в образе возвращения блудного сына\cite[С.~144–145]{krasnoseltcev.uprazdnenniye.1889}.
Женские же песнопения связаны с темами возрастания в добродетели и духовном бодрствовании в образах обручения с Женихом и ожидания Его подобно разумным девам с горящими светильниками:

Первый антифон -- стихира покаянная в неделю вечера на 4-й глас, также поющаяся в неделю о самарянке.
В ней есть образ и покаяния, и невесты: «коль возлюблена села твоя Господи силъ\ldots{} единъ человеколюбецъ Богъ нашъ тя ныне приемлет дево, того всею душею возлюбивши, потому покажи чистое житие. угоди успешно зовящи\ldots{}».
Постригаемая сравнивается с возлюбленной и призывается хранить верность этой любви (ещё пока не браку).
Здесь есть и дерзновенное прошение возлюбленной: «приими мя ходящую во следъ тебе, краснаго жениха моего. не попусти врагу хвалитися мною\ldots{}».

Второй антифон развивает эти образы, становясь как бы иллюстрацией библейской Песни Песней: «Распни уды христолюбивая. умертви страсти воздержаниемъ, и к жениху си теци и взови. где пасеши женише. где живеши рцы ми. яко тебе возлюбивши притекох. на рамо крестъ твой вземши. да вниду в чертогъ свещу носящи, ико дева чистая\ldots{}».
Здесь постригаемая сравнивается с взывающей и ищущей жениха возлюбленной.
К этому яркому, порывистому образу примыкают крест, горящая свеча, чистота, а позже мученица Фекла, риза нетления, Богородица -- образы верности и испытанности.

К третьему антифону постригаемая стоит перед царскими вратами, и в это время поётся: «Господи, отверзи двери чертога рабе твоей, верою припадающи\ldots{}».
Так, образ ищущей чертога жениха в песнопениях дополняется движением к чертогу-алтарю в ходе службы.
В антифоне есть прошения: «сподоби ю всесилне, святаго твоего звания, и спричти ю паствы твоея, Иисусе мой. хотяй спасти всех нас, и отверзи двери правды\ldots{} Отверзи двери правды, вшед в ня\ldots{} Господи, оружие ми подай же на ратоборца. верою и надеждою и любовию укрепи мя всесилне нань яко человеколюбец. побеждьши его съкрушу и поеру, хотяй всех нас спасти».
Здесь о постригаемой молятся как о воине, выходящем на битву с лукавым.

\section*{Шествие под пение антифонов}\label{ux434ux432ux438ux436ux435ux43dux438ux435-ux432ux43e-ux432ux440ux435ux43cux44f-ux430ux43dux442ux438ux444ux43eux43dux43eux432}

Дважды втечение литургии постригаемая движется от притвора к алтарю под пение антифонов.
В разных источниках, её движение заканчивается у царских врат либо уже в алтаре, сперва для пострига, а во второй раз -- для облачения.
Первое движение происходит под антифоны на литургии слова, второе -- на литургии верных.

В этом шествии можно увидеть остатки динамики кафедрального станционного богослужения, когда подобное движение от притвора к амвону и затем к алтарю (а клиром -- в алтарь) совершалось всей паствой.
Там вход в храм совершался на месте нынешнего «малого входа», а к амвону приступали во время пения Трисвятого для слушания Писания и проповеди.
К алтарю приступали на «великом входе» литургии верных.

В чинах постригов эта логика разрушена и остался лишь динамический импульс шествия.
К сожалению, он оторван от смысла собственно литургии и происходит параллельно.
Особенно это заметно в том, что во время чтения постригаемая и сёстры вовсе отсутствуют.
Они в дьяконнике либо в притворе совершают вторую, «братскую» часть пострига и возвращаются лишь с началом литургии верных.

Возможно, этот разрыв богослужебной логики произошёл в связи с многослойностью самого чина пострига, вобравшего, по-видимому, в себя содержание малого и великого постригов, а впоследствие ещё подвергнутого попыткам «свернуть» чин обратно в один из постригов.

\section*{Первый вход в алтарь: постриг}\label{ux432ux445ux43eux434-ux432-ux430ux43bux442ux430ux440ux44c-ux43fux43eux441ux442ux440ux438ux433}

Вход в алтарь тем более интересен, что в настоящее время ни в мужском, ни в женском постриге в великую схиму, постригаемый не вводится в алтарь, если только это не дьякон или священник.

Также интересно, что постриг имеет два этапа: пострижение священником и пострижение сёстрами (по некоторым версиям -- воспреемниками) в дьяконнике или на паперти.
Последнее весьма адекватно чину, т.к. монахиня принимается не только игуменьей, но и сестричеством.

В настоящее время пострижение в великую схиму, подобно пострижению в мантию, не имеет второго этапа пострига.
Ектенья на постриге переходит прямо в ектенью на облачении, а постриг рукой священника -- в облачение тут же, у царских врат.
После того следует обычное завершение чина пострига в великую схиму, а затем -- обычное завершение литургии.
Подобный сокращённый чин великой схимы встречается в Евхологии Афоно--Дионисиатской библ. №450 и предназначен для тех, кто уже прошёл чин малого пострига.

С одной стороны, это выявляет «составную» структуру чина великой схимы, которая, вероятно, сохранилась со времён чина изложенного в Евхологии Барберини св. Марка, где также соединены малая и великая схима.
Таким образом, можно увидеть изначальную внутреннюю логику: постриг соответствовал литургии слова, а облачение -- литургии верных.

Согласно Пальмову, сокращение в Евхологии №450 избавило чин от лишнего, повторяющегося обряда пострига, сохранив лишь обряд облачения, т.к. мантийный монах уже пострижен и его основное отличие -- облачение.
Нельзя не заметить, однако, что сокращение чина выполнено неудачно, с перемещением обряда облачения, логически соответствующего литургии верных, в литургию слова.

Современное же состояние чинов оказывается вовсе неудачным: из малой схимы исчезла вторая часть пострига, а великая схима переместилась целиком в литургию слова.
Таким образом, пострижение братьями (сёстрами) вовсе исчезло из современного чина, а облачение потеряло свой торжественный символизм, связанный с литургией верных.

\section*{Второй вход в алтарь: облачение}\label{ux432ux445ux43eux434-ux432-ux430ux43bux442ux430ux440ux44c-ux43eux431ux43bux430ux447ux435ux43dux438ux435}

По возвращению в церковь после чтения Писания, постригаемая вновь продвигается под пение антифонов к царским вратам.
Далее следует облачение.
По разным источникам оно совершается либо «близ» (видимо, имеется ввиду близ царских врат), либо сказано «вшедше же ей в олтарь».
Последнее встречается в ряд русских источников более раннего происхождения (Требники №998 (XVв.), №1004 (XIVв.)).
Второй из этих требников, при этом, описывает облачение в чине мужского пострига «близ» (Л.139), а женское -- в алтаре (Л.126).
Вероятно, раз разделившись, эти два чина развивались самостоятельно.

Пальмов пишет, что согласно правильному переводу места облачения «τὸ βλῆμα» в греческих Евхологиях, «облачение в великосхимнические одежды необходимо представлять происходящим в алтаре у св. престола»\cite[С.~147]{palmov.postrijeniye.1914}.
