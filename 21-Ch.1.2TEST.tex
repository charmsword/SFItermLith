\section{Аналитические этюды}\label{sec.SmEt}

Проведём анализ открывающий или завершающий сцен каждого фильма, как мест наибольшей силы музыкального воздействия, начиная уже с «классической голливудской» парадигмы саундтрека {[}@\ldots{}{]}.
В случаях, касающихся фильмов, где музыка была особенно отмечена наградами или критикой, мы рассмотрим обе сцены, а также иногда сцену с контрастной, сюжетно значимой музыкальной темой.

Начнём с анализа фильма «Игрок» (1992), так как композитор неоднократно упомянул его, как фильм, в котором он 1) почувствовал смелость всерьёз экспериментировать с музыкой, 2) начал обретать собственный стиль.
Остальные аналитические этюды расположены в хронологическом порядке.
Опущены фильмы меньшей кинематографической значимости.

\subsection{«Игрок» (1992)}\label{sec.SmEt.Pl}

\subsubsection{Общие замечания о фильме}\label{ux43eux431ux449ux438ux435-ux437ux430ux43cux435ux447ux430ux43dux438ux44f-ux43e-ux444ux438ux43bux44cux43cux435}

\subsubsection{Открывающая сцена}\label{ux43eux442ux43aux440ux44bux432ux430ux44eux449ux430ux44f-ux441ux446ux435ux43dux430}

\paragraph{Синопсис сцены}\label{ux441ux438ux43dux43eux43fux441ux438ux441-ux441ux446ux435ux43dux44b}

\paragraph{Описание и интерпретация художественных текстов сцены (визуального, шумового и музыкального)}\label{ux43eux43fux438ux441ux430ux43dux438ux435-ux438-ux438ux43dux442ux435ux440ux43fux440ux435ux442ux430ux446ux438ux44f-ux445ux443ux434ux43eux436ux435ux441ux442ux432ux435ux43dux43dux44bux445-ux442ux435ux43aux441ux442ux43eux432-ux441ux446ux435ux43dux44b-ux432ux438ux437ux443ux430ux43bux44cux43dux43eux433ux43e-ux448ux443ux43cux43eux432ux43eux433ux43e-ux438-ux43cux443ux437ux44bux43aux430ux43bux44cux43dux43eux433ux43e}

\paragraph{Интерпретация роли музыки сцены в общей композиции фильма}\label{ux438ux43dux442ux435ux440ux43fux440ux435ux442ux430ux446ux438ux44f-ux440ux43eux43bux438-ux43cux443ux437ux44bux43aux438-ux441ux446ux435ux43dux44b-ux432-ux43eux431ux449ux435ux439-ux43aux43eux43cux43fux43eux437ux438ux446ux438ux438-ux444ux438ux43bux44cux43cux430}

\subsection{«Плоть от плоти» (1993)}\label{sec.SmEt.FnB}

\subsubsection{Общие замечания о фильме}\label{ux43eux431ux449ux438ux435-ux437ux430ux43cux435ux447ux430ux43dux438ux44f-ux43e-ux444ux438ux43bux44cux43cux435-1}

Фильм «Плоть от плоти» написан и срежиссирован в 1993 г. американским сценаристом Стивом Кловсом (Steve Kloves).
В фильме собрана блестящая команда актёров (Мэг Райн, Деннис Куэйд, Джеймс Каан и молодая Гвинет Пэлтроу в одной из своих первых ролей), но кассовые сборы были невелики, как и оценка ленты критикой.
Саундтрек к фильму стал одной из ранних работ Томаса Ньюмана, где зарождались особенности его собственного стиля.

Фильм назван по жанру «нео-нуаровым», но «нуаровых» черт в нём очень немного.
Можно назвать лишь: 1) характер главного героя с его замкнутостью, подавленной эмоциональностью, грубоватой, строгой мужественностью, 2) прошлое, безжалостно преследующее и настигающее главных героев, 3) «изнанка» общества, в которой живут главные герои, не тяготеющие к активным социальным ролям, а ищущих себе «складки», где бы укрыться, 4) романтическая линия, не имеющая шанса на успех по причине рокового прошлого героев.
В остальном, история развивается на фоне американской сельской местности, и простор неба и прерий в ней контрастирует с грязью и убожеством жизни в мотелях, захолустных городках, ночных клубах.
Последняя не романтизируется, как это происходит в фильмах нуар, где даже понятие «грязь» эстетически обыгрывается, а начинает рифмоваться с неожиданными коннотациями красивых пейзажей: их безлюдностью, пустынностью, бедностью, пугающей открытостью.
Таким образом, вместо нео-нуара, фильм представляет, скорее, американскую «готическую» драму, где на фоне прерий южных штатов рассказываются истории в духе готических и пред-готических романов XIX века.

В музыке Ньюману, таким образом, пришлось изобретать музыкальный ряд для этого необычного кинематографического жанра.
Он создал саундтрек, в котором можно выделить два типа композиций: пугающий, зловещий эмбиент, часто с конкретными звуками или подражанием им, и неоромантический, с интонациями американской фолк-музыки.
Но, как всегда, композиции богаты на необычные приёмы инструментовки и стилистические отклонения.\\
Композитор упоминает, что саундтрек к «Плоти от плоти» -- одна из значимых для него ранних работ, где ему удался экспериментальный подход к инструментовке: сочетание «атмосферных (`ambient') звуков, конкретных и конкретно-инструментальных звуков (`found sound') и сэмплированных звуков со струнным оркестром»\footnote{«I did a movie called `Flesh and Bone' back in 1991 that I thought was a breakthrough score for me in how I combined ambient sounds, found sounds and sampled sounds with a string orchestra. It was a real breakthrough for me conceptually» \autocite{TN.flesh.2012}.}.

\subsubsection{Открывающая сцена}\label{ux43eux442ux43aux440ux44bux432ux430ux44eux449ux430ux44f-ux441ux446ux435ux43dux430-1}

\paragraph{Синопсис сцены}\label{ux441ux438ux43dux43eux43fux441ux438ux441-ux441ux446ux435ux43dux44b-1}

\begin{longtable}[]{@{}ll@{}}
\toprule
\begin{minipage}[b]{0.29\columnwidth}\raggedright\strut
кадр\strut
\end{minipage} & \begin{minipage}[b]{0.65\columnwidth}\raggedright\strut
комментарий\strut
\end{minipage}\tabularnewline
\midrule
\endhead
\begin{minipage}[t]{0.29\columnwidth}\raggedright\strut
1:00
\includegraphics{IMG/FnB_1-00.png}\strut
\end{minipage} & \begin{minipage}[t]{0.65\columnwidth}\raggedright\strut
Титры с именами актёров на чёрном, сумеречном пейзаже
американской сельской глубинки, на фоне догорающего заката.
Громко стрекочут цикады.\strut
\end{minipage}\tabularnewline
\begin{minipage}[t]{0.29\columnwidth}\raggedright\strut
\begin{figure}
\centering
\includegraphics{Ly/SmEt/FnBly_Hinge_CROP.png}
\caption{}
\end{figure}
\strut
\end{minipage} & \begin{minipage}[t]{0.65\columnwidth}\raggedright\strut
Сонорный эмбиент тибетских чаш и скрежетов, шелестов и лязга гонга.\strut
\end{minipage}\tabularnewline
\begin{minipage}[t]{0.29\columnwidth}\raggedright\strut
2:05
\includegraphics{IMG/FnB_2-05.png}\strut
\end{minipage} & \begin{minipage}[t]{0.65\columnwidth}\raggedright\strut
Вокруг дома пусто и тихо, недвижно висят качели, стоят кресла,
спит собака.\strut
\end{minipage}\tabularnewline
\begin{minipage}[t]{0.29\columnwidth}\raggedright\strut
диегическая музыка\strut
\end{minipage} & \begin{minipage}[t]{0.65\columnwidth}\raggedright\strut
Из телевизора доносится драматичная музыка, пока внутри дома
супруга не просит мужа выключить его, чтобы не мешать детям спать.\strut
\end{minipage}\tabularnewline
\begin{minipage}[t]{0.29\columnwidth}\raggedright\strut
2:51--11:23
\includegraphics{IMG/FnB_2-51.png}\strut
\end{minipage} & \begin{minipage}[t]{0.65\columnwidth}\raggedright\strut
Следует долгая сцена завязки фильма: обитатели дома встречают под дверью
одинокого мальчика, Арлиса, принимают его, кормят и укладывают спать.
Ночью Арлис открывает входную дверь и внутрь прокрадывается его отец с ружьём.
Когда их застигают за воровством, отец убивает хозяев дома, кроме маленькой
девочки в колыбельке.\strut
\end{minipage}\tabularnewline
\begin{minipage}[t]{0.29\columnwidth}\raggedright\strut
пауза\strut
\end{minipage} & \begin{minipage}[t]{0.65\columnwidth}\raggedright\strut
\strut
\end{minipage}\tabularnewline
\begin{minipage}[t]{0.48\columnwidth}\raggedright\strut
11:23

\begin{figure}
\centering
\includegraphics{IMG/FnB_11-23.png}
\caption{}
\end{figure}
\strut
\end{minipage} & \begin{minipage}[t]{0.48\columnwidth}\raggedright\strut
Завязка завершается кадром с юным Арлисом и его отцом в темноте
чужого дома. Плачет ребёнок. Над домом занимается рассвет.\strut
\end{minipage}\tabularnewline
\begin{minipage}[t]{0.29\columnwidth}\raggedright\strut
пауза\strut
\end{minipage} & \begin{minipage}[t]{0.65\columnwidth}\raggedright\strut
\strut
\end{minipage}\tabularnewline
\begin{minipage}[t]{0.29\columnwidth}\raggedright\strut
12:20
\includegraphics{IMG/FnB_12-20.png}\strut
\end{minipage} & \begin{minipage}[t]{0.65\columnwidth}\raggedright\strut
День. Небольшой городок в южных штатах. Взрослый Арлис за рулём автомобиля.\strut
\end{minipage}\tabularnewline
\begin{minipage}[t]{0.29\columnwidth}\raggedright\strut
\begin{figure}
\centering
\includegraphics{Ly/SmEt/FnBly_BD_A_CROP.png}
\caption{}
\end{figure}
\strut
\end{minipage} & \begin{minipage}[t]{0.65\columnwidth}\raggedright\strut
Композиция в стилистике фолк и кантри, с гитарой, мандолиной, скрипкой и
экмелическими «завываниями» этнической флейты.\strut
\end{minipage}\tabularnewline
\bottomrule
\end{longtable}

\paragraph{Описание и интерпретация художественных текстов сцены (визуального, шумового и музыкального)}\label{ux43eux43fux438ux441ux430ux43dux438ux435-ux438-ux438ux43dux442ux435ux440ux43fux440ux435ux442ux430ux446ux438ux44f-ux445ux443ux434ux43eux436ux435ux441ux442ux432ux435ux43dux43dux44bux445-ux442ux435ux43aux441ux442ux43eux432-ux441ux446ux435ux43dux44b-ux432ux438ux437ux443ux430ux43bux44cux43dux43eux433ux43e-ux448ux443ux43cux43eux432ux43eux433ux43e-ux438-ux43cux443ux437ux44bux43aux430ux43bux44cux43dux43eux433ux43e-1}

Фильм открывается долгим, 12-минутным вступлением.
В течение 2 минут звучит вступительная тема Ньюмана, \emph{Hinge}, затем плавно расстворяющаяся в диегической музыке из телевизора и тихих ночных звуках, окружающих сельский дом.
После драматической, насыщенной 10-минутной завязки, начинается первый акт фильма и вступает вторая музыкальная тема, \emph{Blue Dimes}.

\emph{Hinge} вступает на первом вступительном титре «Paramount Pictures presents», когда экран погас и появляются первые визуальные знаки фильма: шероховатый, словно бы стёртый, с щербинками, шрифт, дающий имена актёров и название фильма прописными буквами.
Это характерный ещё со времени готического романа эффект пугающей, тревожной старины, с коннотациями заброшенного, оставленного людьми, но некогда связанного с людьми.
Далее этот знак будет развиваться в кадрах: в оставленных тракторе, повозке, лязгающей ветряной мельницей, пустых качелях, пустых креслах, спящей собаке.
Над долгими, вдумчивыми кадрами работал блестящий оператор Филипп Русло (Philippe Rousselot), и визуальный ряд здесь продуман и поэтически, и драматически.
Предметы здесь показаны живущими своей жизнью, без нужды в человеческих хозяевах, чем усиливается эффект пугающей безлюдности места.
Тревожный догорающий закат и громкое стрекотание цикад добавляет страха в эти, на первый взгляд, безобидные кадры сельской местности.

\includegraphics{Ly/SmEt/FnBly_Hinge.png}
\includegraphics{IMG/FnB_Collage_2.jpg}

Ньюман поэтически отражает эти знаки и настроения в музыке.
Название «\emph{Hinge}» означает `скрежет'.
Композиция представляет сонорное поле из равномерно повторяющихся d--B у тибетских чаш, через долгий, полный обертонов отзвук которых время от времени мерцает отдалённый металлический скрежет и лязг -- это движение смычка по гонгу и проведение металлической палочки по его ребру.
С одной стороны, это рифмы с диегизисом сцены: с лязгом, скрипами, позвякиванием мельницы, качелей, подвешенных на крыльце ложек.
С другой, это развитие настроения, задаваемого предметами в кадре через включение их звуков в музыкальную композицию с такими усиливающими приёмами как остинатное повторение, полифония голосов, большие тембровые возможности\footnote{Так, можно отметить, что ни один конкретный звук из кадра не способен создать такого атмосферного скрежета как игра смычком по большому гонгу.}.

Также, звуки \emph{Hinge} имеют собственные коннотации: тембр тибетских чаш с остинатным, медленным повторением звуков, мгновенно создаёт коннотации с древностью и с восточной ритуальностью.
Древность рифмуется со «стёртыми» шрифтами и безлюдными пейзажами.
Восточная ритуальность характерна для пантеистической или буддистской культуры, с характерными мотивами «расстворения» человека в духовном.
Звук чаш, предназначенных для медитации, погружает в себя, отделяет от привычного, за что человек держится, и в сочетании с отдалёнными «диегическими» скрежетами и лязгами, зритель оказывается затем целиком погружён в ночную сцену в доме.

Эффект затем усиливается звучанием телевизора из дома, который доносится на крыльцо в виде обрывков драматической, даже пугающей музыки.
\emph{Hinge} уже смолкла, но незаметно, «расстворившись» в скрипе качелей и звоне подвешенных ложек, а потому возникает диссонанс двух музыкальных пластов, а с ним -- когнитивный диссонанс двух миров, «дневного человеческого» и «ночного безлюдного».
Этот диссонанс становится саспенсом, напряжением, которое требует разрешения и в течение всей последующей сцены в доме, несмотря на обилие света и бытовые разговоры, зритель ожидает чего-то жуткого.
Когда свет уходит из сцены, мир ночи из открывающих кадров вновь всплывает, и подчёркивается звоном больших напольных часов, отбивающих полночь.
Сцена убийств проходит без музыки.

Дальше происходит смена сцен с контрастом в музыке и визуальном ряду: ночь сменяется днём, испуганный мальчик -- решительным угрюмым мужчиной, его бессильный взгляд в сторону -- уверенными движениями водителя автомобиля, и долгую паузу сменяет \emph{Blue Dimes} (рис. \ref{img.smet.fnbly.bd.a}).

\begin{figure}
\centering
\includegraphics{Ly/SmEt/FnBly_BD_A.png}
\caption{}\label{img.smet.fnbly.bd.a}
\end{figure}

С одной стороны, это -- разрешение когнитивного диссонанса между тишиной, звуковой безответстностью предыдущих кадров по отношению к трагедии.
Зритель непременно эмоционально реагирует на события в ночной сцене, но не находит сочувствия в музыке.
Визуальный и драматический ряды отдаляются здесь от музыкального по причине «отмалчивания» последнего.

Здесь же музыка решительно вступает, и она сочувствует и совпадает с происходящем на экране, что позволяет зрителю «выдохнуть»: диссонанс визуального и музыкального разрешён.
При этом, он разрешён в позитивную сторону: звучит музыка в духе американского фолка с характерным сочетанием лихости гитары и мандолины и мужества, серьёзности и даже печали скрипки (рис. \ref{img.smet.fnbly.bd.b}).
Это «рифмуется» и с обликом взрослого Арлиса, с его «ковбойским» видом и с сеттингом южных штатов.
Это и явно противопоставляется \emph{Hinge}, так как здесь звучит чувственная, сопереживающая, эмоциональная музыка, впервые близкая и приближающая к людям на экране, а не отдаляющая от них.

\begin{figure}
\centering
\includegraphics{Ly/SmEt/FnBly_BD_B.png}
\caption{}\label{img.smet.fnbly.bd.b}
\end{figure}

С другой стороны, романтическую \emph{Blue Dimes} то и дело прорезает экмелический взвизг «флейты» -- некоего этнического или электронного инструмента (рис. \ref{img.smet.fnbly.bd.a}, такты 2, 3).
Звук реверберирован и повисает в воздухе, да ещё и задерживается на диссонантном f' (по отношению к fis в e-дорийском) и затем «сползает» к VII ступене d'.
Этот вскрик добавляет в музыку элемент «дикости» чересчур сильный даже по меркам кантри и фолка, где привычны различные «присвисты», прихлопывания и вскрики.
В результате, звук становится рифмой к прошлой сцене, которая не до конца забывается и расстворяется в дневном свете.
Фактически, это музыкальная метафора «расколотости» жизни взрослого Арсля -- с виду собранной и даже привлекательной, а внутри с надломом и страхом прошлого.
В то же время, надлом здесь подавляется дневным светом и активной позицией персонажа, и потому добавляет скорее «соли» в музыкальный ряд, не даёт ему быть пресным и индиффирентным по отношению к событиям за пределами насущного кадра.

\paragraph{Интерпретация роли музыки сцены в общей композиции фильма}\label{ux438ux43dux442ux435ux440ux43fux440ux435ux442ux430ux446ux438ux44f-ux440ux43eux43bux438-ux43cux443ux437ux44bux43aux438-ux441ux446ux435ux43dux44b-ux432-ux43eux431ux449ux435ux439-ux43aux43eux43cux43fux43eux437ux438ux446ux438ux438-ux444ux438ux43bux44cux43cux430-1}

Музыка в открывающей сцене и на стыке пролога и первого акта фильма рифмуется с когнитивным диссонансом изображения, денотативного наполнения кадров и, наконец, диссонансом драматическим: трагедией в жизни героя, преследующим его роковым прошлым.
Эти сильные образы мифологические образы выражены через сопоставление «несовместимых» музыкальных пластов: медитативной, атмосферной и страшной \emph{Hinge} и лихой, романтической \emph{Blue Dimes}.
Это эклектическое сочетание, подобное сопоставлению несовместимых музыкальных пластов в музыке Ч. Айвза.
Ни в жанре медитативной музыки, ни в жанре эмбиента нет места для скрипок и мандолин \emph{Blue Dimes}, а в фолке, которому подражает последняя нет места «брейкам» с такими настроениями.
Здесь рождается своеобразный синтетический стиль, который имеет образные сферы, адекватно выражаемые в таких разных стилях и их сочетаниях и, при этом, не «разваливаться».
Этому помогает то, что сопоставление несочетаемых стилей само по себе оказывается художественным выражением вопиющего, безобразного прошлого и его отражений в настоящем, и желания персонажей избавиться от него.

Это близко к эстетике \emph{магического реализма}, где «прорыв» несовместимого «ночного» мира в повседневность художественно может выражаться «микстом сельской и городской культур, архаики и авангарда, подстмодернизма» \autocite[ см. \emph{Магический реализм}]{bychkov.lexikon.2003}.
Особую роль здесь играет поэтизация денотативного, а не только коннотативного содержания сюжета: «второстепенные» детали, их положение, форма, материал, из оформления становятся метафорой, приобретают собственную коннотацию.
Это происходит в кадре, это подхватывает музыка, продлевая в себе денотативные звук и пространство кадра, материал декорации, сообщая им поэтические свойства музыкального материала.
Это -- также черта \emph{магического реализма}, не «копирующего реальность (как делали реалисты) и не ``взвинчивающего'' её (как делали сюрреалисты), но ищущего тайну, что дышит за каждым предметом\ldots{}», для которого «принципиально не столько создавать волшебные предметы и воображаемые миры, сколько раскрыть таинственную связь человека и окружающих его обстоятельств» \autocite[ пер. -- мой]{carpentier.magicalrealism.1995}.
В «Плоти от плоти» основное выражение трагического надлома и противопоставления чистоты и «грязи» (порочности, печати преступления) происходит в поэтическом раскрытии денотативного, т.е. «окружающих человека обстоятельств», тайны «каждого предмета».

Оказывается, что «потёртые», полузаброшенные пейзажи прерий и городов южных штатов США становится столь же подходящим сеттингом готического повествования, каким в XIX в. были европейские средневековые или античные руины.
Здесь Ньюману пришлось изобретать музыкальный язык, соответствующий этому эклектическому, \emph{магико-реалистическому} стилю, и он пришёл к созданию столь же эклектичного саундтрека.
Композиционная целостность саундтрека при этом сохраняется и благодаря уже привычной современному зрителю-слушателю пёстрой стилистики \emph{магического реализма} и постмодернизма в целом, и благодаря взаимопроникновению элементов контрастных композиций друг в друга: экмелики и сонористики в «фолк» и «кантри», романтических мотивов (пусть и в деконструированном, разрушенном виде) -- в сонористические, атмосферные «эмбиенты».

\subsection{«Побег из Шоушенка» (1994)}\label{sec.SmEt.Sh}

\subsubsection{Общие замечания о фильме}\label{ux43eux431ux449ux438ux435-ux437ux430ux43cux435ux447ux430ux43dux438ux44f-ux43e-ux444ux438ux43bux44cux43cux435-2}

Томас Ньюман здесь почти целиком обращается к симфоническому оркестру, отказываясь от традиционного для себя «экзотического» звука.

\subsubsection{Открывающая сцена}\label{ux43eux442ux43aux440ux44bux432ux430ux44eux449ux430ux44f-ux441ux446ux435ux43dux430-2}

\paragraph{Синопсис сцены}\label{ux441ux438ux43dux43eux43fux441ux438ux441-ux441ux446ux435ux43dux44b-2}

\begin{longtable}[]{@{}ll@{}}
\toprule
\begin{minipage}[b]{0.30\columnwidth}\raggedright\strut
кадр\strut
\end{minipage} & \begin{minipage}[b]{0.64\columnwidth}\raggedright\strut
комментарий\strut
\end{minipage}\tabularnewline
\midrule
\endhead
\begin{minipage}[t]{0.30\columnwidth}\raggedright\strut
8:51 \textbackslash{}
\includegraphics{IMG/Sh_8-51.png}\strut
\end{minipage} & \begin{minipage}[t]{0.64\columnwidth}\raggedright\strut
РЭД и другие заключённые идут по тюремному двору.
Неожиданно звучит сирена, они оборачиваются на звук.\strut
\end{minipage}\tabularnewline
\begin{minipage}[t]{0.30\columnwidth}\raggedright\strut
\begin{figure}
\centering
\includegraphics{Ly/SmEt/SmEt/tacet.png}
\caption{}
\end{figure}
\strut
\end{minipage} & \begin{minipage}[t]{0.64\columnwidth}\raggedright\strut
\strut
\end{minipage}\tabularnewline
\begin{minipage}[t]{0.30\columnwidth}\raggedright\strut
9:01 \textbackslash{}
\includegraphics{IMG/Sh_9-01.png}\strut
\end{minipage} & \begin{minipage}[t]{0.64\columnwidth}\raggedright\strut
Автобус с заключёнными заворачивает к тюрьме.
Камера следует за ним, затем обгоняет автобус и
устремляется к зданию тюрьмы.\strut
\end{minipage}\tabularnewline
\begin{minipage}[t]{0.30\columnwidth}\raggedright\strut
\begin{figure}
\centering
\includegraphics{Ly/SmEt/Shly_9-01.png}
\caption{}
\end{figure}
\strut
\end{minipage} & \begin{minipage}[t]{0.64\columnwidth}\raggedright\strut
Начинается остинатная фигура у контрабасов.\strut
\end{minipage}\tabularnewline
\begin{minipage}[t]{0.30\columnwidth}\raggedright\strut
9:16 \textbackslash{}
\includegraphics{IMG/Sh_9-16.png}\strut
\end{minipage} & \begin{minipage}[t]{0.64\columnwidth}\raggedright\strut
Камера проносится близко к тёмным крышам тюремного
комплекса и над внутренним двором, усеянным людьми,
стекающимся к воротам.\strut
\end{minipage}\tabularnewline
\begin{minipage}[t]{0.30\columnwidth}\raggedright\strut
\begin{figure}
\centering
\includegraphics{Ly/SmEt/Shly_9-16.png}
\caption{}
\end{figure}
\strut
\end{minipage} & \begin{minipage}[t]{0.64\columnwidth}\raggedright\strut
Из остинато вырастает фугато, равномерное по ритму,
изломанное диссонантными сочетаниями, но со светлым оттенком
с-дорийского.\strut
\end{minipage}\tabularnewline
\begin{minipage}[t]{0.30\columnwidth}\raggedright\strut
9:37 \textbackslash{}
\includegraphics{IMG/Sh_9-37.png}\strut
\end{minipage} & \begin{minipage}[t]{0.64\columnwidth}\raggedright\strut
Единственное цветное пятно на блеклом пейзаже тюремного
двора с серыми фигурами заключённых -- синий флаг с гербом
тюрьмы, реющий на ветру.\strut
\end{minipage}\tabularnewline
\begin{minipage}[t]{0.30\columnwidth}\raggedright\strut
\begin{figure}
\centering
\includegraphics{Ly/SmEt/Shly_9-37.png}
\caption{}
\end{figure}
\strut
\end{minipage} & \begin{minipage}[t]{0.64\columnwidth}\raggedright\strut
Появляются тёплые тембры валторн и краска параллельного
Es-dur.\strut
\end{minipage}\tabularnewline
\begin{minipage}[t]{0.30\columnwidth}\raggedright\strut
9:48 \textbackslash{}
\includegraphics{IMG/Sh_9-48.png}\strut
\end{minipage} & \begin{minipage}[t]{0.64\columnwidth}\raggedright\strut
Внутри тюремного автобуса, на дальнем сидении сидит
угрюмый, забывшийся в мыслях ЭНДИ.
РЭД за кадром коротко рассказывает его историю.\strut
\end{minipage}\tabularnewline
\begin{minipage}[t]{0.30\columnwidth}\raggedright\strut
\begin{figure}
\centering
\includegraphics{Ly/SmEt/Shly_9-48.png}
\caption{}
\end{figure}
\strut
\end{minipage} & \begin{minipage}[t]{0.64\columnwidth}\raggedright\strut
Остинатная фигура исчезает, звучание более светлое,
тихое.\strut
\end{minipage}\tabularnewline
\begin{minipage}[t]{0.30\columnwidth}\raggedright\strut
10:00 \textbackslash{}
\includegraphics{IMG/Sh_10-00.png}\strut
\end{minipage} & \begin{minipage}[t]{0.64\columnwidth}\raggedright\strut
Бегущие через двор заключённые возбуждены, слышатся
апплодисменты, свист. РЭД узнаёт кого-то в толпе, улыбается,
подаёт знак рукой и бежит встретиться.\strut
\end{minipage}\tabularnewline
\begin{minipage}[t]{0.30\columnwidth}\raggedright\strut
\begin{figure}
\centering
\includegraphics{Ly/SmEt/Shly_10-00.png}
\caption{}
\end{figure}
\strut
\end{minipage} & \begin{minipage}[t]{0.64\columnwidth}\raggedright\strut
Лиричное звучание держится недолго, потом вновь «придавлено»
басами.\strut
\end{minipage}\tabularnewline
\begin{minipage}[t]{0.48\columnwidth}\raggedright\strut
10:16 \textbackslash{}

\begin{figure}
\centering
\includegraphics{IMG/Sh_10-16.png}
\caption{}
\end{figure}
\strut
\end{minipage} & \begin{minipage}[t]{0.48\columnwidth}\raggedright\strut
Узкий коридор тюремных заборов преграждён охранниками, среди
выделяется высокая фигура БАЙРОНА ХЭДЛИ, начальника охраны,
жестом командующего автобусу подъезжать ближе.\strut
\end{minipage}\tabularnewline
\begin{minipage}[t]{0.30\columnwidth}\raggedright\strut
\begin{figure}
\centering
\includegraphics{Ly/SmEt/Shly_10-16.png}
\caption{}
\end{figure}
\strut
\end{minipage} & \begin{minipage}[t]{0.64\columnwidth}\raggedright\strut
Утверждается низкий регистр: дробь литавр, контрабасы,
тромбоны.
В гармонии резкие противопоставления: неожиданный C --\textgreater{}
Es(9, 13) --\textgreater{} DesSus4 --\textgreater{} Bsus2; а в мелодии ламентозная
интонация c'--\textgreater{}des'--\textgreater{}c'.
Ладовая краска «темнеет» до фригийско-локрийской с низкой
второй и пятой ступенями.\strut
\end{minipage}\tabularnewline
\begin{minipage}[t]{0.48\columnwidth}\raggedright\strut
10:22 \textbackslash{}

\begin{figure}
\centering
\includegraphics{IMG/Sh_10-22.png}
\caption{}
\end{figure}
\strut
\end{minipage} & \begin{minipage}[t]{0.48\columnwidth}\raggedright\strut
Охранники с автоматами сбегают с вышки, чтобы занять
позицию над воротами. Внизу вокруг забора толпятся
заключённые, кричат и колотят по металлической сетке.\strut
\end{minipage}\tabularnewline
\begin{minipage}[t]{0.30\columnwidth}\raggedright\strut
\begin{figure}
\centering
\includegraphics{Ly/SmEt/Shly_10-22.png}
\caption{}
\end{figure}
\strut
\end{minipage} & \begin{minipage}[t]{0.64\columnwidth}\raggedright\strut
Под дробь литавр струнная и медная секции
охватывают весь регистр, особенно яркие на краях:
в линии басов и ламентозной мелодии скрипок.\strut
\end{minipage}\tabularnewline
\begin{minipage}[t]{0.48\columnwidth}\raggedright\strut
10:27 \textbackslash{}

\begin{figure}
\centering
\includegraphics{IMG/Sh_10-27.png}
\caption{}
\end{figure}
\strut
\end{minipage} & \begin{minipage}[t]{0.48\columnwidth}\raggedright\strut
ХЭДЛИ заглядывает в автобус, оттуда начинают
по очереди выводить заключённых.\strut
\end{minipage}\tabularnewline
\begin{minipage}[t]{0.30\columnwidth}\raggedright\strut
\begin{figure}
\centering
\includegraphics{Ly/SmEt/Shly_9-01.png}
\caption{}
\end{figure}
\strut
\end{minipage} & \begin{minipage}[t]{0.64\columnwidth}\raggedright\strut
Звучание быстро «сворачивается» в остинато контрабасов,
и вскоре музыка смолкает.\strut
\end{minipage}\tabularnewline
\bottomrule
\end{longtable}

\paragraph{Описание и интерпретация художественных текстов сцены (визуального, шумового и музыкального)}\label{ux43eux43fux438ux441ux430ux43dux438ux435-ux438-ux438ux43dux442ux435ux440ux43fux440ux435ux442ux430ux446ux438ux44f-ux445ux443ux434ux43eux436ux435ux441ux442ux432ux435ux43dux43dux44bux445-ux442ux435ux43aux441ux442ux43eux432-ux441ux446ux435ux43dux44b-ux432ux438ux437ux443ux430ux43bux44cux43dux43eux433ux43e-ux448ux443ux43cux43eux432ux43eux433ux43e-ux438-ux43cux443ux437ux44bux43aux430ux43bux44cux43dux43eux433ux43e-2}

Темп и равномерность движения в музыке целиком совпадают с движением камеры: сперва следованием за автобусом и медленным «пролётом» над башнями и крышей тюрьмы и внутренним двором, затем -- вслед за Рэдом в толпе, медленным приближением камеры к Энди в автобусе.
Движение камеры здесь не на миг не застывает и кажется равномерным благодаря разнице в масштабе кадров, несмотря на то, что вертолёт с оператором облетает тюрьму на скорости, а внутри автобуса камера движется очень медленно.
Это движение камеры маскирует монтажные стыки и создаёт впечатление одного долгого кадра, полифонически раскрывающего сцену с разных сторон: Рэда на тюремном дворе, Энди в автобусе, всего тюремного мира с высоты птичьего полёта (или сторожевой вышки).
Полифонически же\footnote{Точнее, псевдо-полифонически, т.к. здесь нет контрапунктно развивающейся темы, а есть обрастающая мотивами в разных голосах фактура, что характерно для пост-минимализма.} развивается музыка: на остинатный мотив контрабасов нанизываются голоса струнных.

Ещё одна рифма текстов фильма -- мрачная краска низких струнных и блеклые цвета и, особенно, серый и чёрный в мощных башнях и широкой крыше тюрьмы.
Приземистая, но величественная как романский собор, тюрьма, медленное движение камеры и автобуса, сам тюремный автобус, «кургузый», крепкий, -- всё это создаёт ощущение массы, тяжеловесности, так отчётливо проступающей и в низкочастотном тембре контрабасов.
Эта метонимическая рифма становится метафорой давящей несвободы, тюрьмы.

С пятого же такта темы начинается контрапункт -- два голоса скрипок, в коротких полифонических мотивах подчёркивающие дорийский ля-бекар.
Эта светлая краска «рифмуется» с неожиданным обилием неба и света в кадрах тюрьмы.
Она экспрессивна, постоянно с ламентозным движением a--b--a или диссонатными интервалами между этими нотами, отличаясь от равномерной поступи басов внутренним движением к развитию.

Вместе с планом камеры меняется фактура оркестра: камерное, тихое звучание и узкие интервалы, когда камера берёт средний план в автобусе.
Здесь Es-лидийский удачно совпадает с мягким голосом Моргана Фримена, в роли Рэда рассказывающего за кадром историю Энди.

Когда сквозь эту толпу движется Рэд, улыбаясь и делая знак рукой кому-то невидимому зрителю, возникает образ ликования, торжества, с которым причудливо перекликаются апплодисменты и свист арестантов.
Атмосфера торжественности становится мрачной, угрожающей, но по-прежнему вооружённая охрана на вышках и даже антагонист, Байрон Хэдли, вписываются в величественное, словно бы эпическое повествование.
Театральный жест Хэдли, манящего автобус пальцем, совпадает хоралом гармоний хроматического родства, а выбегающие с вышки на стену охранники с пулемётами, -- с тремолло литавр, аккордами медной группы и расширением текстуры струнных до масштабного диапазона от низких нот контрабаса к 1й октаве скрипок.

\paragraph{Интерпретация роли музыки сцены в общей композиции фильма}\label{ux438ux43dux442ux435ux440ux43fux440ux435ux442ux430ux446ux438ux44f-ux440ux43eux43bux438-ux43cux443ux437ux44bux43aux438-ux441ux446ux435ux43dux44b-ux432-ux43eux431ux449ux435ux439-ux43aux43eux43cux43fux43eux437ux438ux446ux438ux438-ux444ux438ux43bux44cux43cux430-2}

Композитор здесь извлекает максимум возможностей из дорийского лада: используя и его светлую краску мажорной субдоминанты и параллельного Es-лидийского, и тёмные оттенки малой секунды и большой септимы a--b.
Через это музыка выражает и драматический излом в жизни персонажа, и драму несвободы огромного количества людей в тюрьме, и в то же время образ «полёта» и движения.
Последний рифмуется с полётом и движением камеры, а вместе они становятся метафоной какой-то скрытой надежды, не подавленности персонажей под гнётом действительности.
Здесь, в отличии от типичных способов изображения тюрьмы, в кадре есть изобилие воздуха, неба, движения.
Противопоставленные этому кадры внутри автобуса только подчёркивают эту «светлую» краску, отзывающуюся у скрипок в дорийском ладу.
Таким образом, создаётся (поздне-)романтический образ тюрьмы, не реалистический и не экспрессионистический.
В этой сцене, таким образом, закладывается стилистика повествования и основной «пружины» сюжета: внутренней негаснущей надежды Энди на возвращение к свободной жизни.
Альтернативное название этой композиции -- «стоическая тема» (`stoic theme', англ.), но это уже следствие романтизации стоицизма в современной популярной культуре, так как ядро у неё романтическое, эмоциональное и чувственное.

В сцене эстетизируются, романтизируются даже такие мрачные образы как толпа заключённых, бегущих глазеть на новоприбывших.
Этому способствуют такие приёмы как противопоставление мрачного c-фригийско-локрийского на появлении Байрона Хэдли и торжественного, неожиданного C-ионийского.
То, что должно пугать в жизни (вооружённая охрана, скопление арестантов, массивное здание тюрьмы), здесь впечатляет, захватывает дух, создаёт эстетический эффект \emph{возвышенного}.

В дальнейшем в фильме образ тюрьмы будет создан в отношениях персонажей и между персонажами в большей степени, чем в чувственных образах карцеров и давящих стен.
Несвобода главного героя будет чувствоваться в постоянной угрозе со стороны арестантов, а позже -- охраны и тюремного начальства.
Это позволяет создать интересный художественный эффект: ликование от освобождения героя зритель чувствует не столько в смене интерьеров тюрьмы экстерьерами природы и города, сколько в снятии этой угрозы.
Таким образом, стилистика романтизма здесь имеет и черты реализма: уже не столько внешные, природные знаки становятся символами внутренней жизни, сколько от зрителя требуется психологически сопереживать герою, чтобы всерьёз понять главную коллизию свободы/несвободы в произведении.

\subsection{«Зелёная миля» (1999)}\label{sec.SmEt.GM}

\subsubsection{Открывающая сцена}\label{ux43eux442ux43aux440ux44bux432ux430ux44eux449ux430ux44f-ux441ux446ux435ux43dux430-3}

\paragraph{Синопсис сцены:}\label{ux441ux438ux43dux43eux43fux441ux438ux441-ux441ux446ux435ux43dux44b-3}

\begin{quote}
Чёрный экран с титрами сменяется кадрами хлопкового поля.
Мы находимся среди растений.
Мимо в замедленной съёмке пробегают один за другим люди с ружьями, одетые как фермеры южных штатов США первой половины ХХ века.
Один из них оказывается рядом с нами и находит клочок одежды на стебле хлопка.
Его лицо искажено отчаянием, он оборачивается назад и кричит что-то, затем устремляется вперёд.
В ту же сторону следом за ним бежит множество мужчин, вооружённых ружьями и вилами.\\
\textbf{Наплыв}\\
\textbf{Тёмный экран с заглавием фильма: «Зелёная миля»}\\
Слышится чей-то злой шёпот: «Если ты и твоя сестрёнка будут молчать, ничего плохого не случится\ldots{}»\\
\textbf{Наплыв}\\
\textbf{ОП:} Глаза старика, только что в испуге проснушегося.
\end{quote}

\paragraph{Описание и интерпретация визуального, шумового и музыкального текстов сцены}\label{ux43eux43fux438ux441ux430ux43dux438ux435-ux438-ux438ux43dux442ux435ux440ux43fux440ux435ux442ux430ux446ux438ux44f-ux432ux438ux437ux443ux430ux43bux44cux43dux43eux433ux43e-ux448ux443ux43cux43eux432ux43eux433ux43e-ux438-ux43cux443ux437ux44bux43aux430ux43bux44cux43dux43eux433ux43e-ux442ux435ux43aux441ux442ux43eux432-ux441ux446ux435ux43dux44b}

Фильм открывается чрезвычайно эмоциональной сценой.
Первый же визуальный образ, человек с ружьём в руках, бегущий среди хлопковых стеблей, обещает историю о насилии, возможно убийстве и смерти.
Далее уровень тревожности и ожиданий чего-то пугающего повышается: 1) следом следуют другие вооружённые люди, что означает погоню или бегство, 2) крупный план лица одного из мужчин и его крик, 3) в кадре появляется целый отряд беспорядочно вооружённых людей.

Но, эмоциональная буря и знаки агрессии оттеняются тут же двумя статическими знаками: 1) чрезвычайно замедленное движение, в котором бег преследователей вызывает больше ассоциаций с движениями пловцов или танцоров, 2) умиротворяющий вид залитых золотистым солнечным светом хлопковых растений, среди которых «проплывают» люди.

Всё же, если увидеть сцену без музыки, в ней будет гораздо больше тревоги или даже ужаса.
Музыка здесь подчёркивает статические элементы визуального ряда и в результате, создаёт атмосферу воспоминания или сна, а также трагизма и горя в большей степени чем страха.

Ещё во время заставки начинается музыка, под которую темнеет экран, а затем появляется первый кадр фильма.
Это -- сонорное поле, колеблющееся вокруг D5 и Hdim6, с повторяющимся у разных инструментов внутри поля восхождением: a--h--c--d.
Эта краска настолько искусно создана, что уже трудно определить, задействованы ли электронные инструменты, или использовались только акустические, но с необычными техниками исполнения, непривычным составом и фоноколористическими техниками.
Вероятно, основу составляет пэд, к которому примешиваются звуки вибрафона, струнной секции.
Тембр пэда несколько напоминает хор и орган, звучащие издалека, что создаёт призрачный образ голосов, а также акустический образ большого пространства.
Таким образом, поле хлопка и люди на нём обретают свои «голоса».

Это -- статическая часть.
Драматическое же развитие музыки в сцене достигается, помимо тембрового добавления к пэду, «акцентами»: 1) глухим низким ударом, 2) высокими глиссандирующими на малую секунду струнными.
Первое -- это низкочастотный удар с долгим отзвуком, наподобие звука, который получается при ударе по педали сустейна рояля с открытой крышкой.
Второе -- это «стоны» скрипки, звучащей очень далеко, играющей sul pontichello «сползания» нот h--b и g--a.
Вместе они создают ритмический рисунок, который делит тему на фрагменты и придаёт ей динамику:

\begin{figure}
\centering
\includegraphics{Ly/smet/GM_field.png}
\caption{}
\end{figure}

С этим музыкальным образом интересно сочетаются шумы, которые также в этой сцене художественно отобраны и обработаны.
Голоса людей не слышны, они остаются с музыкальным «голосом».
Крик мужчины обработан так, что от него осталось лишь отдалённое, приглушенное эхо.
Но, это эхо во-первых содержит в себе остатки безумной силы и экспрессии первоначального звука.
Тем страшнее оно звучит «приглушенное», «сдавленное», так как лицо человека показано крупным планом и психоакустическое воображение зрителя невольно «достраивает» этот крик.
Во-вторых, крик поддержан музыкальным «ударом».

Когда вооружённые люди устремляются вместе с кричавшим в едином порыве, музыкальная текстура концентрируется на отчётливом B-dur.
Остановка на гармонии двойной доминанты создаёт ещё большее напряжение и ожидание развязки.
Но, развязки нет в визуальном ряду: люди продолжают бежать всё быстрее, пока кадр не гаснет.
В музыке текстура D5--Hdim6 неожиданно разрешается в новую текстуру-звучность.

Новое сонорное поле окрашено мелодическим A-dur.
Оно ощущается как более «прозрачное», по фактуре: в нём нет ни высоких «стонов» скрипок и низких ударов, и по гармонии: на место бесконечного хроматического, диссонантного движения к разрешению, звучит диатоника.
Эта «прозрачность» утверждается новым тембром, исполняющим неточно повторяющиеся, аритмичные фигурации, раскачивающиеся по нотам мелодического A-dur от A к G.
Это, вероятно, составной тембр из фортепиано и ксилофона.
«Деревянная» звучность и необычная, как бы не контролируемая человеком партия, вызывают ассоциации с «музыкой ветра»\footnote{«Музыкой ветра» называют разнообразные народные музыкальные инструменты в виде связок деревянных или металлических дощечек, колокольчиков, камешков или иных предметов, свободно свисающих с ветвей деревьев, «козырьков» у дверей и иных мест, где ветер, играя ими и заставляя их стучать друг о друга, создаёт «музыку».}.

Вместе с гаснущим экраном, смолкают почти все элементы текстуры, остаётся только интервал у струнных: e--a.
На нём, на фоне чёрного экрана и названия фильма, слышатся отдалённые крики мужчины, появлявшегося в кадре крупным планом, он произносит два женских имени.
Когда стихает последнее музыкальное созвучие и гаснет название фильма, раздаётся напряжённый шёпот неизвестного мужского персонажа: «Ты любишь сестрёнку? Пикнешь -- знаешь, что будет»\footnote{`You love your sister? You make any noise and you know what happens'.}.
Кадр резко сменяется сверхкрупным планом резко открывшихся, слезящихся старческих глаз.
Слышится звук работающего телевизора и напряжённое дыхание только что проснувшегося старика.

\paragraph{Интерпретация роли музыкального оформления сцены}\label{ux438ux43dux442ux435ux440ux43fux440ux435ux442ux430ux446ux438ux44f-ux440ux43eux43bux438-ux43cux443ux437ux44bux43aux430ux43bux44cux43dux43eux433ux43e-ux43eux444ux43eux440ux43cux43bux435ux43dux438ux44f-ux441ux446ux435ux43dux44b}

В первой, б\_о\_льшей части сцены, музыка играет двойную роль: музыкального и звукового оформления.
Гулкие низкие удары и «стоны» скрипки, похожие на крики птиц или плач детей, настолько переплетаются с диегическими звуками сцены (отдалённым лаем собак и криком мужчины, чьим-то напряжённым дыханием), что составляют единую художественную текстуру.

Пэд и отдельные элементы в звуковом поле, представляют сонорную звучность, не привлекающую к себе внимания как к музыке, а представляющей эмбиент.
В результате отсутствия реальных шумов в сцене, которая должна бы быть залита звуками природы и погони, и в результате соответствия статики музыки и замедленной съёмки, здесь также происходит замещение в уме зрителя.
Невольно музыка воспринимается на месте отсутствующего звукового оформления.
Звуковое оформление сцены, в свою очередь, снято с привычного места и в результате фоноколористической обработки, вписано в музыку.

Здесь все три текста фильма звучат слитно.
Это не разрушается даже тем, что музыка вступает первой, до первого кадра сцены.
Их соединяет общее настроение статики, но с большим внутренним напряжением-диссонансом.
Диссонантное искажение в хроматическом ходу в музыке сливается с диссонантом искажённого звука крика и искажённого лица человека в кадре, а также трагическим противопоставлением в кадре: красоты золотого хлопкового поля летним днём и вооружённой, отчаянной погони.

Но дальше происходит расхождение музыки с визуальным и шумовым текстами.
Музыка становится явно закадровой.
Во-первых, она в один момент меняется, хотя кадр продолжается.
Во-вторых, она очень серьёзно меняет свой характер, что при внешне мало приметных изменениях фактуры (в основном остаётся тот же приём статического сонорного поля пэда с мелодико-гармоническими «колебаниями» внутри), производит эффект смыслового акцента.
Зритель отчётливо слышит в этой перемене некоторое высказывание, коннотацию.
Если до того музыка не была заметна и представляла, фактически, денотацию, здесь музыка становится заметной и появляется явная коннотативная отсылка к какому-то явлению или персонажу, совершенно незнакомому пока зрителю.
Это вынуждает зрителя «гадать», исходя из того, о чём говорит новая, заметная музыка.

Вступившая тема, во-первых, в основе мажорна и диатонична.
Во-вторых, в ней присутствуют сочетания гармоний хроматического родства: это и само её вступление на A-dur после «неаполитанской» B-dur (затихающей ещё очень долго), и намёк через мелодические f--a на F-dur.
В фактуре сочетаются три очень характерных тембра.
Внизу -- басовая педаль низких струнных, что при хорошей ревербирации, вызывает образ мужского хора, а значит строгости, суровости и силы.
В верхнем регистре продолжает звучать отзвук B-dur у пэда с выделенными средними частотами, что также вынуждает его звучать похоже на человеческий голос, теперь уже либо на отзвуки обертонов отдалённого мужского хора, либо на женский хор.
Акустическое пространство огромное, что вместе с хоровым пением создаёт образ торжественный, даже священный.

Но, внутри себя этот образ имеет гротескные черты.
Во-первых, это краски гармоний третьей степени родства.
Трудно представить аутентичную религиозную музыку до ХХ века, в которой было бы возможно хроматическое сопоставление гармоний.
Однако, вполне можно представить (и даже неоднократно слышать), как это сопоставление образуется в отзвуках или искажениях звука отражением (выделении каких-то обертонов акустикой помещения) или в случайном наложении двух разных музыкальных пластов (например, двух хоров, поющих поблизости) или непредумышленной фальши исполнителей.
Здесь же именно это, случайно возникающее в аутентичной музыке, маргинальное явление, и вызываемые им чувства, которые обычно отметаются слушателем ради основного содержания пения, берутся за основу.
Это напоминает Чарльза Айвза, вероятно, первым увидевшего музыкальный смысл в этих случайных сочетаниях и начавшего из коллажа денотативных текстов создавать новые музыкальные коннотации {[}@\ldots{}{]}.

Другая гротескная черта -- соединение этой «строгой» звучности (а также трагической строгости предыдущей музыкальной текстуры) с самовольным «бряцанием» «музыки ветра» (точнее, её имитацией у фортепиано и ксилофона).
Последняя создаёт смысловой диссонанс и жанра, и стиля, и даже психоакустический.
Психоакустически -- мы никогда бы не услышали перезвон «музыки ветра» так отчётливо на фоне мощных хоров.
Здесь же возникает эффект «сочетания несочетаемых» музыкальных пластов: либо зритель слышит эту «музыку ветра» «внутри» себя, либо он прислушивается именно к ней, уйдя хоров, потому столь далёких.
То есть, он опять прислушивается к маргиналиям.
И «хоры» звучат в маргинальном виде -- столь же важным как их тембр и гармония, оказывается факт их отдаления и художественный эффект, порой воспроизводимый в оперной или симфонической музыкой игрой инструмента или группы инструментов за кулисами.
Такой приём всегда оставался «спецэффектом», здесь же это -- одна из красок музыкальной палитры, «эмансипированная» наравне с гармонией и фактурой.

В дальнейшем эта «тема» станет лейтмотивом магического, некоторого актанта, постоянно противопоставляемого реалистической манере повествования.
Этот актант присутствует и в стилистики оригинального произведения, книги Стивена Кинга, лёгшей в основу фильма.
Таким образом фильм, вслед за книгой, представляет жанр магического реализма, в котором реальность и реалистическая стилистика то и дело оборачивается паранормальной «изнанкой».
Эта потусторонняя реальность становится одной из движущих сил сюжета: своим вторжением (в лице арестанта Джона Коффи) она меняет течение жизни других героев фильма.
Образ этой пара-реальности -- гротескное искажение образов реальности или сюрреалистическое сопоставление «несочетаемых» элементов.

Так, эта лейттема появляется в кадрах явления чудесных целительных сил Джона Коффи.
Точнее, она сопутствует единственным кадрам в фильме со специальным, нарисованным

\paragraph{Таблица}\label{ux442ux430ux431ux43bux438ux446ux430}

\begin{longtable}[]{@{}ll@{}}
\caption{Открывающая сцена «Зелёной мили» (0:31--2:07)}\tabularnewline
\toprule
\begin{minipage}[b]{0.30\columnwidth}\raggedright\strut
кадр\strut
\end{minipage} & \begin{minipage}[b]{0.55\columnwidth}\raggedright\strut
комментарий\strut
\end{minipage}\tabularnewline
\midrule
\endfirsthead
\toprule
\begin{minipage}[b]{0.30\columnwidth}\raggedright\strut
кадр\strut
\end{minipage} & \begin{minipage}[b]{0.55\columnwidth}\raggedright\strut
комментарий\strut
\end{minipage}\tabularnewline
\midrule
\endhead
\begin{minipage}[t]{0.30\columnwidth}\raggedright\strut
\begin{figure}
\centering
\includegraphics{IMG/GM_0.31.png}
\caption{}
\end{figure}
\strut
\end{minipage} & \begin{minipage}[t]{0.55\columnwidth}\raggedright\strut
заставка, чёрный экран\strut
\end{minipage}\tabularnewline
\begin{minipage}[t]{0.30\columnwidth}\raggedright\strut
0:31\strut
\end{minipage} & \begin{minipage}[t]{0.55\columnwidth}\raggedright\strut
тревожный эмбиент\strut
\end{minipage}\tabularnewline
\begin{minipage}[t]{0.30\columnwidth}\raggedright\strut
\begin{figure}
\centering
\includegraphics{IMG/GM_0.50.png}
\caption{}
\end{figure}
\strut
\end{minipage} & \begin{minipage}[t]{0.55\columnwidth}\raggedright\strut
поле хлопка,
бегущие вооружённые люди\strut
\end{minipage}\tabularnewline
\begin{minipage}[t]{0.30\columnwidth}\raggedright\strut
0:50\strut
\end{minipage} & \begin{minipage}[t]{0.55\columnwidth}\raggedright\strut
тревожный эмбиент с «криками чаек»\strut
\end{minipage}\tabularnewline
\begin{minipage}[t]{0.30\columnwidth}\raggedright\strut
\begin{figure}
\centering
\includegraphics{IMG/GM_1.20.png}
\caption{}
\end{figure}
\strut
\end{minipage} & \begin{minipage}[t]{0.55\columnwidth}\raggedright\strut
КП: мужчина с отчаянием на лице,
находит лоскут одежды, кричит\strut
\end{minipage}\tabularnewline
\begin{minipage}[t]{0.30\columnwidth}\raggedright\strut
1:20\strut
\end{minipage} & \begin{minipage}[t]{0.55\columnwidth}\raggedright\strut
---``---\strut
\end{minipage}\tabularnewline
\begin{minipage}[t]{0.30\columnwidth}\raggedright\strut
\begin{figure}
\centering
\includegraphics{IMG/GM_1.42.png}
\caption{}
\end{figure}
\strut
\end{minipage} & \begin{minipage}[t]{0.55\columnwidth}\raggedright\strut
ОП: много мужчин с вилами и ружьями
бегут вместе, всё быстрее\strut
\end{minipage}\tabularnewline
\begin{minipage}[t]{0.30\columnwidth}\raggedright\strut
1:42\strut
\end{minipage} & \begin{minipage}[t]{0.55\columnwidth}\raggedright\strut
пэд-доминанта\strut
\end{minipage}\tabularnewline
\begin{minipage}[t]{0.30\columnwidth}\raggedright\strut
\begin{figure}
\centering
\includegraphics{IMG/GM_1.45.png}
\caption{}
\end{figure}
\strut
\end{minipage} & \begin{minipage}[t]{0.55\columnwidth}\raggedright\strut
---``---\strut
\end{minipage}\tabularnewline
\begin{minipage}[t]{0.30\columnwidth}\raggedright\strut
1:45\strut
\end{minipage} & \begin{minipage}[t]{0.55\columnwidth}\raggedright\strut
разрешение в «магическую» тему\strut
\end{minipage}\tabularnewline
\begin{minipage}[t]{0.30\columnwidth}\raggedright\strut
\begin{figure}
\centering
\includegraphics{IMG/GM_1.50.png}
\caption{}
\end{figure}
\strut
\end{minipage} & \begin{minipage}[t]{0.55\columnwidth}\raggedright\strut
наплыв: название фильма на чёрном фоне\strut
\end{minipage}\tabularnewline
\begin{minipage}[t]{0.30\columnwidth}\raggedright\strut
1:50\strut
\end{minipage} & \begin{minipage}[t]{0.55\columnwidth}\raggedright\strut
разрешение\strut
\end{minipage}\tabularnewline
\begin{minipage}[t]{0.30\columnwidth}\raggedright\strut
\begin{figure}
\centering
\includegraphics{IMG/GM_2.05.png}
\caption{}
\end{figure}
\strut
\end{minipage} & \begin{minipage}[t]{0.55\columnwidth}\raggedright\strut
CКП: глаза пробудившегося старика\strut
\end{minipage}\tabularnewline
\begin{minipage}[t]{0.30\columnwidth}\raggedright\strut
2:05\strut
\end{minipage} & \begin{minipage}[t]{0.55\columnwidth}\raggedright\strut
пауза, угрожающий шепот из воспоминаний\strut
\end{minipage}\tabularnewline
\bottomrule
\end{longtable}

\subsubsection{Сцена-экспозиция главных героев фильма}\label{ux441ux446ux435ux43dux430-ux44dux43aux441ux43fux43eux437ux438ux446ux438ux44f-ux433ux43bux430ux432ux43dux44bux445-ux433ux435ux440ux43eux435ux432-ux444ux438ux43bux44cux43cux430}

\paragraph{Синопсис музыкальной части сцены}\label{ux441ux438ux43dux43eux43fux441ux438ux441-ux43cux443ux437ux44bux43aux430ux43bux44cux43dux43eux439-ux447ux430ux441ux442ux438-ux441ux446ux435ux43dux44b}

\begin{longtable}[]{@{}ll@{}}
\toprule
\begin{minipage}[b]{0.27\columnwidth}\raggedright\strut
кадр\strut
\end{minipage} & \begin{minipage}[b]{0.68\columnwidth}\raggedright\strut
комментарий\strut
\end{minipage}\tabularnewline
\midrule
\endhead
\begin{minipage}[t]{0.27\columnwidth}\raggedright\strut
12:14 \textbackslash{}
\includegraphics{IMG/GM_12-14.png}\strut
\end{minipage} & \begin{minipage}[t]{0.68\columnwidth}\raggedright\strut
ОХРАННИКИ «Мили» ожидают прибытия арестанта.
ПОЛ идёт к свободной камере, БРУТ -- к двери,\strut
\end{minipage}\tabularnewline
\begin{minipage}[t]{0.27\columnwidth}\raggedright\strut
\strut
\end{minipage} & \begin{minipage}[t]{0.68\columnwidth}\raggedright\strut
\strut
\end{minipage}\tabularnewline
\begin{minipage}[t]{0.27\columnwidth}\raggedright\strut
12:16--18 \textbackslash{}
\includegraphics{IMG/GM_12-18.jpg}\strut
\end{minipage} & \begin{minipage}[t]{0.68\columnwidth}\raggedright\strut
БРУТ смотрит в окошко бронированной двери и удивлённо
щурится: что-то не так.
Тюремный фургон въезжает во двор, сильно оседая на задние
колёса.\strut
\end{minipage}\tabularnewline
\begin{minipage}[t]{0.27\columnwidth}\raggedright\strut
\strut
\end{minipage} & \begin{minipage}[t]{0.68\columnwidth}\raggedright\strut
\strut
\end{minipage}\tabularnewline
\begin{minipage}[t]{0.27\columnwidth}\raggedright\strut
12:20--31 \textbackslash{}
\includegraphics{IMG/GM_12-20.jpg}\strut
\end{minipage} & \begin{minipage}[t]{0.68\columnwidth}\raggedright\strut
ПЕРСИ обходит фургон и открывает замок на его дверях.
Двери со скрежетом раскрываются, оттуда выходят
двое вооружённых охранников.
БРУТ: Что они там натворили? Сломали армотизаторы?\strut
\end{minipage}\tabularnewline
\begin{minipage}[t]{0.27\columnwidth}\raggedright\strut
12:32--39 \textbackslash{}
\includegraphics{IMG/GM_12-32.jpg}\strut
\end{minipage} & \begin{minipage}[t]{0.68\columnwidth}\raggedright\strut
Из фургона показываются две огромные ступни в кандалах,
это чернокожий арестант ДЖОН КОФФИ. Когда он спрыгивает
с подножки, фургон подскакивает на рессорах.
БРУТ изумлённо восклицает.\strut
\end{minipage}\tabularnewline
\begin{minipage}[t]{0.48\columnwidth}\raggedright\strut
12:45 \textbackslash{}

\begin{figure}
\centering
\includegraphics{IMG/GM_12-45.png}
\caption{}
\end{figure}
\strut
\end{minipage} & \begin{minipage}[t]{0.48\columnwidth}\raggedright\strut
ПЕРСИ ведёт КОФФИ, выкрикивая: «Вот идёт мертвец! Глядите,
покойник идёт!».
ПОЛ: (раздражённо): Чего он там орёт?\strut
\end{minipage}\tabularnewline
\begin{minipage}[t]{0.27\columnwidth}\raggedright\strut
12:58 \textbackslash{}
\includegraphics{IMG/GM_12-58.png}\strut
\end{minipage} & \begin{minipage}[t]{0.68\columnwidth}\raggedright\strut
БРУТ: Пол, тебе не захочется заходить с ним в камеру.
ПОЛ: Почему же?
БРУТ: Он огромен!
ПОЛ: Ну, не больше тебя?
БРУТ ухмыляется -- сейчас сам поглядишь.\strut
\end{minipage}\tabularnewline
\begin{minipage}[t]{0.27\columnwidth}\raggedright\strut
13:16 \textbackslash{}
\includegraphics{IMG/GM_13-16.png}\strut
\end{minipage} & \begin{minipage}[t]{0.68\columnwidth}\raggedright\strut
БРУТ распахивает дверь блока, внутрь врываются солнечный
свет и выкрики ПЕРСИ.
ПОЛ закатывает глаза в раздражении.\strut
\end{minipage}\tabularnewline
\begin{minipage}[t]{0.27\columnwidth}\raggedright\strut
13:22--28 \textbackslash{}
\includegraphics{IMG/GM_13-22.jpg}\strut
\end{minipage} & \begin{minipage}[t]{0.68\columnwidth}\raggedright\strut
В ПРОХОДЕ появляется спева ПЕРСИ, за ним -- огромная фигура
КОФФИ, заслонившего дневной свет. Лица, не видно, оно
за верхней границей кадра.\strut
\end{minipage}\tabularnewline
\begin{minipage}[t]{0.27\columnwidth}\raggedright\strut
13:32--52 \textbackslash{}
\includegraphics{IMG/GM_13-32.jpg}\strut
\end{minipage} & \begin{minipage}[t]{0.68\columnwidth}\raggedright\strut
ПЕРСИ, не переставая кричать, ведёт КОФФИ через блок,
назидательно указывая на него дубинкой другим заключённым.
Тень от проходящего ДЖОН КОФФИ падает на выглянувших на
крик арестантов.\strut
\end{minipage}\tabularnewline
\begin{minipage}[t]{0.27\columnwidth}\raggedright\strut
13:58 \textbackslash{}
\includegraphics{IMG/GM_13-58.jpg}\strut
\end{minipage} & \begin{minipage}[t]{0.68\columnwidth}\raggedright\strut
ПОЛ осаживает ПЕРСИ, тот недовольно замолкает.
ПОЛ нерешительно смотрит на огромного арестанта.
БРУТ не скрывает испуга при взгляде на мышцы рук КОФФИ,
размером с дыню.\strut
\end{minipage}\tabularnewline
\begin{minipage}[t]{0.48\columnwidth}\raggedright\strut
14:05--11 \textbackslash{}

\begin{figure}
\centering
\includegraphics{IMG/GM_14-05.jpg}
\caption{}
\end{figure}
\strut
\end{minipage} & \begin{minipage}[t]{0.48\columnwidth}\raggedright\strut
ПОЛ: (с деловым видом, вздохнув) У меня будут с тобой
проблемы, здоровяк?
КОФФИ молчит.\strut
\end{minipage}\tabularnewline
\begin{minipage}[t]{0.27\columnwidth}\raggedright\strut
14:21 \textbackslash{}
\includegraphics{IMG/GM_14-21.png}\strut
\end{minipage} & \begin{minipage}[t]{0.68\columnwidth}\raggedright\strut
ПОЛ: (невольно улыбаясь) Умеешь разговаривать?
КОФФИ: (глядя сверху вниз, потерянный) Да, начальник.
Я умею.\strut
\end{minipage}\tabularnewline
\bottomrule
\end{longtable}

\begin{verbatim}
          Brutal

          peers out the viewing slot as the truck stops outside.

                              BRUTAL
                    Damn, they're riding on the axle.
                    What'd they do, bust the springs?

          GUARDS PERCY WETMORE AND HARRY TERWILLIGER OF E BLOCK emerge
          from the back of the truck and step down, turn back...

          Tighter angle on back of truck

          We get our first glimpse of the new inmate as a pair of
          GIGANTIC BLACK FEET step down into the yard...and the rear of
          the truck bounces back up on its springs where it belongs.

          Brutal

          sees what's coming, eyes widening slightly.

                              BRUTAL
                    Paul? You might wanna reconsider
                    getting in the cell with this guy?

                              PAUL
                    Why's that?

                              BRUTAL
                    He's enormous.

                              PAUL
                    Can't be bigger than you.

          Brutal tosses him a look--just wait. He swings the door open
          in a hot flood of daylight, giving us our first good look at:

          John coffey

          is a huge black man, nearly 7 feet tall and 300 pounds, his
          massive head shiny and bald, his skin a tapestry of old
          scars, his prison overalls (the biggest size they had) ending
          at mid-calf. He looks dull and confused, as if wondering
          where he is and how he got there. Percy and Harry lead him
          toward E Block in shackles. Percy's got his hickory baton out
          of it custom-made holster, hollering:

                              PERCY
                    Dead man walking! Dead man walking
                    here!

          Inside the cellblock

          Paul can't see them approach from where he stands, but he can
          certainly hear Percy:

                              PAUL
                    Jeezus, pleeze-us, what the hell's
                    he yelling about?

          Up by the door, Brutal just rolls his eyes. Percy is the
          first one through the door, still hollering...

                              PERCY
                    Dead man walking!

          ...then Coffey enters, ducking low to get through, his shadow
          blotting out Brutal and Dean as his massive frame fills the
          door. Everything hangs suspended for a moment, a look of
          "hold shit" written on everybody's faces. Percy keeps yanking
          on the big man's cuffs, leading him along with a cry of:

                              PERCY
                    Dead man walking! Dead man--

                              PAUL
                    Percy, that's enough.

          Percy falls reproachfully silent. Paul doesn't dignify it,
          just motions for them to come on. The procession comes down
          the Mile, with Brutal and Dean bringing up rear.

                              BRUTAL
                    You sure you wanna be in there
                    with him?

                              PAUL
                           (looks to Coffey)
                    Am I gonna have trouble with you,
                    big boy?

          Coffey shakes his head slowly. Paul takes the clipboard
          transfer papers from Harry, turns and enters the cell.
\end{verbatim}

\paragraph{Описание и интепретация визуального, шумового и музыкального текстов сцены}\label{ux43eux43fux438ux441ux430ux43dux438ux435-ux438-ux438ux43dux442ux435ux43fux440ux435ux442ux430ux446ux438ux44f-ux432ux438ux437ux443ux430ux43bux44cux43dux43eux433ux43e-ux448ux443ux43cux43eux432ux43eux433ux43e-ux438-ux43cux443ux437ux44bux43aux430ux43bux44cux43dux43eux433ux43e-ux442ux435ux43aux441ux442ux43eux432-ux441ux446ux435ux43dux44b}

В этой сцене в кадре и обретают некоторую характеристику значительная часть персонажей фильма: почти вся команда охранников, включая протагониста ПОЛА ЭДЖКОМБА и антагониста ПЕРСИ, а также заключённый ДЖОН КОФФИ и двое других арестантов в камерах смертников.

Сцена эмоционально очень нагружена, причём спектр эмоций широкий: от комичного первоначального изумления охранников перед ростом ДЖОНА КОФФИ до трепета и суеверного испуга, внушаемого исполином.
А контрапунктом к этому -- вызывающий мгновенную неприязнь зрителя Перси, с удовольствием выкрикивающего свою фразу о мертвеце, каждый раз придумывая ей какую-нибудь новую вариацию.
В визуальном ряду первое и последнее читается в лицах персонажей, благодаря талантливой актёрской игре: полуулыбки, с которыми переглядываются ПОЛ и БРУТУС и выражение наглой жестокости ПЕРСИ.

Второе, эмоциональный образ трепета и страха, выражен явными визуальными гиперболами, едва ли не гротескно изображающими исполинские масштабы ДЖОНА КОФФИ.
Это: 1) оседающий и подпрыгивающий под его весом фургончик для арестантов, 2) кадры, на которых фигура КОФФИ «обрезана» до плечей, так что до диалога с ПОЛОМ мы видим только его огромные руки в кандалах, 3) движение КОФФИ по коридору обозначено также кадрами, в которых его громадная тень накрывает одного за другим выглядывающих из камер двух арестантов.

Сцена начинается со звука клаксона, возвестившего охранникам о прибытие арестанта.
Музыка начинается до появления в кадре ДЖОНА КОФФИ, и первая возвещает, что происходит нечто существенное для сюжета.
Не считая вступления, закадровая музыка в фильме до этого момента прозвучала только раз, напрямую соответствуя содержанию кадра: лирическая \emph{the Green Mile} аккомпанировала прогулке пожилого ПОЛА ЭДЖКОМБА по полю и лесу в пасмурный день.
С того момента (5:09) закадровой музыки не было уже семь минут (до 12:12), несмотря на смену множества кадров.
Кроме этого, музыка сразу даёт о себе знать несоответствием кадру: реалистической обстановке охранников, совершающих формальную, хоть и несколько напряжённую, процедуру доставки заключённого в блок, аккомпанирует неожиданно «мистически» окрашенный звук: тембр пэда, напоминающий хор, бессловесно поющий в малой октаве модально перетекающие друг в друга гармонии g-moll--Es-dur.
Этот музыкальный знак «разгадывается» зрителем, когда из фургона появляются две огромные, в кандалах, ступни чернокожего арестанта.

Здесь происходит своеобразный эллипсис: едва успел разрешиться когнитивный диссонанс между кадром и музыкой, как тут же визуальная гипербола снова поражает воображение зрителя, -- фургон подскакивает на рессонах, когда КОФФИ спрыгивает с подножки.
Сцена как бы на миг вернулась в реалистическую манеру (стало ясно, что загадочная музыка связана с необычайной силой арестанта, что, конечно, опасно для охранников), но тут же «выпрыгнула» в гротеск: заключённый не просто сильный, большой и опасный, он -- нечеловечески, непостижимо огромен.

Музыка подхватывает и подчёркивает этот гротеск двумя знаками.
Это «денотативная» низкая нота у струнных, вынуждающая зрителя буквально физически ощутить дрожь земли, когда КОФФИ выходит из фургона, и это неожиданная по стилистике фигурация гитары «Добро» с блюзовыми фаршлагами, что вызывает ассоциацию со стилем кантри и фолк музыки.
С одной стороны, это неожиданное для ситуации стилевое наклонение в сторону музыки, которя ассоциируется с удалью, лихостью, но никак не с реалистически-серьёзной сценой встречи опасного заключённого.
С другой -- это знак, соответствующий сеттингу, штату Алабама 1930-х гг., который потому не звучит нарочито чуждым.
Его «лихая» интонация, остинатно повторяясь каждый такт, вместе с низкими струнными, создаёт образ торжественного, но фантасмагорического шествия.

Когда БРУТУС открывает дверь тюрьмы, чтобы впустить Коффи, музыкальный период завершается на доминанте, вновь через далёкую «неаполитанскую» ступень.
Со входом ПЕРСИ и КОФФИ, образ величия и трепета ещё разрастается.
Охранники показаны в кадре, задирающими головы в изумлении и попытки оглянуть всю фигуру исполина, в то время как струнные ведут вверх широкими интервалами фразу, сперва завершившуюся на c'`, а затем забирающуюся до b''.
C-миксолидийский здесь «откликается» сопоставлением C-dur и Fis-dur.
Это эмоциональная кульминация сцены.

Дальше визуальная гипербола и музыкальный гротеск постепенно «смягчаются», дав несколько последних акцентов виде тени КОФФИ, накрывающий фигуры арестантов и затихающих у струнных красок C-лидийского, Fis-миксолидийского.
Сцена возвращается в реалистическое повествование через тривиальные действия: ПОЛ осаживает ПЕРСИ, затем, по протоколу, вводит арестанта в камеру, представляется, объясняет правила и т.д..
Здесь вновь величие, трепет и первенство эмоциональной стороны уступает место рациональному мышлению зрителя: опасный заключённый в одной камере с охранником, заключённый оказывается послушным, ПЕРСИ оказывается негодяем, но с ним можно справиться и т.д..

\paragraph{Интерпретация роли музыкального оформления сцены}\label{ux438ux43dux442ux435ux440ux43fux440ux435ux442ux430ux446ux438ux44f-ux440ux43eux43bux438-ux43cux443ux437ux44bux43aux430ux43bux44cux43dux43eux433ux43e-ux43eux444ux43eux440ux43cux43bux435ux43dux438ux44f-ux441ux446ux435ux43dux44b-1}

В сцене музыка играет значительную роль, подчёркивая когнитивный диссонанс, заложенный в драматургию фильма \emph{магическим реализмом} первоисточника.
Реалистическое в основе своей повествование «расщепляется» на фантасмагорию и реальность, и зрителю невозможно их различить.
Так как в жизни фантазия и реальность, как правило, различаются людьми, такой образ создаёт напряжение, диссонанс, характерный для \emph{гротеска}.
Этот диссонанс подчёркнут экзотическими сочетаниями инструментов, стилистики и гармонии.

Сочетание у струнных и гитары C-лидийского (CM(+4) и F-миксолидийского с квартой (F7(4,-7)) вызывает краску одновременно гротескную внутренней диссонантностью аккордов с большими и малыми секундами и проступающим через них противопоставлением далёких C-dur и es-moll.
Фигура КОФФИ, обрезанная кадром до плечей (при этом, охранники на его фоне всегда попадают в кадр целиком, даже высокого роста БРУТУС), соответствует этому в визуальном ряду.
Зритель не видит лица арестанта и пытается представить лицо, соответствующее такой могучей фигуре.
Это придаёт происходящему, при всей его карнавальности, атмосферу серьёзности, которая сохраняется помимо всех «комичных» элементов.
Здесь юмор не разоблачает силу, как часто бывает в сатире (например, в «марше Черномора» М. И. Глинки или в гротесках А. Г. Шнитке), а подчёркивает её (как это происходит, например, у Мусоргского в «Песнях и плясках Смерти»).
Образ величия и трепета становится ещё более непостижимым, нечеловеческим, когда звучит с контрапунктом в стиле кантри: так невольные взгляды и улыбки, которыми обмениваются охранники блока, не невилируют, а подчёркивают для зрителя их испуг и изумление.

Этот образ неоднократно будет развиваться и, что характерно, в следующий раз интонации восходящих скрипок, обозначавшие исполинский рост ДЖОНА КОФФИ, прозвучат на одном из редких общих планов: кадре тюрьмы, освещённой ударами молний в грозовую ночь перед казнью одного из арестантов (1:37:00).

В дальнейшем в течение фильма когнитивный диссонанс фантасмагории и реальности подчеркнёт другой, ещё более важный эстетический диссонанс: между трагическим и комическим.
В основе своей фильм трагичен и рассказывает о гибели и даже самопожертвовании невинного человека.
При этом, этот человек, ДЖОН КОФФИ в ходе фильма оказывается проводником чудесных исцелений в жизни героев, они находят в нём друга и пытаются спасти, но безуспешно.
Эта сюжетная линия могла бы быть чересчур сентиментальной и даже надрывной, если бы не оттенение её юмором бытовых ситуаций, смекалки протагонистов, простой, но остроумной шутке КОФФИ и т.д..

Вокруг одного неожиданного персонажа развивается целая сфера «несерьёзной» драматургии и сооветствствующей музыки в стилистике кантри и фолка: вокруг мыши по имени «мистер Джингллс».
Этот зверёк -- характерный пример, как образы фильма «глядят» в обе стороны: комедии и трагедии, возвышенного и карнавального.
Он пробуждает негодяя ПЕРСИ раскрыться с ещё большей злобой: тот ставит себе цель уничтожить зверька.
Но, через него же ПЕРСИ выставлен нелепым и истеричным, когда под лихие наигрыши гитары и банджо неожиданно с криком, напугавшим других охранников, начинает бросать в зверька предметы и пытается его растоптать.
Мышонок становится ручным зверьком одного из заключённых и показывает фокусы, иллюстрируемые забавной и карнавальной \emph{Circus Mouse} с экмелическими выкриками флейт.
Но он же становится одним из знаков трагедии и смерти, когда арестант передаёт его перед своей казнью ПОЛУ ЭДЖКОМБУ.
«Мистер Джингллс» гибнет под каблуком ПЕРСИ и оживает под действием невероятных целительных сил КОФФИ, совершившего так своё второе чудо.
В конце фильма он становится уже образом потустороннего, непостижимого мира, так как его мышиный век растянулся на шестьдесят лет, и ему сопутствует уже не комичная музыка, но перекаты гармоний хроматического родства.

Подобным образом, необычно и неожиданно изображён арестант БИЛЛ УАРТОН.
Это закоренелый преступник, несмотря на свой молодой возраст, единственный по-настоящему опасный в блоке смертников.
Один раз он чуть не задушил охранника ДИНА, в качестве шуточной попытки побега, неоднократно издевался над охранниками и другими заключёнными и, наконец, в конце выясняется, что он стоит за жутким преступлением, из-за которого на электрический стул идёт КОФФИ.
Визуально, его сцены уродливы, как его выходки, -- он творит очередную бесстыдную «шутку» и с упоением хохочет, колотя себя ладонями по бёдрам.
Но, всегда эти сцены имеют в себе комическую нотку: либо в самой его выходке, либо в справедливом и, подчас, довольно остроумном возмездии за это.
Музыкальная характеристика БИЛЛА дана через кантри и фолк мотивы гитар, банджо, варгана, что придаёт ему вид бандита с Дикого Запада.
Его «зловещая» сторона не выражена в музыке.
В итоге, по ходу фильма зрителю становится страшно не от вида безобразного преступника и ожидания новых бедствий от него, а от совершающейся трагедии, от гибели ДЖОНА КОФФИ.

Таким образом, эта сцена-экспозиция и музыка в ней, помимо экспонирования персонажей, задают тон всему дальнейшему повествованию, в ней уже заложены знаки основных художественных сфер, через которых будут нарисованы все дальнейшие персонажи и события фильма.

\subsection{«Красота по-Американски» (1999)}\label{sec.SmEt.AB}

\subsubsection{Общие замечания о фильме}\label{ux43eux431ux449ux438ux435-ux437ux430ux43cux435ux447ux430ux43dux438ux44f-ux43e-ux444ux438ux43bux44cux43cux435-3}

\subsubsection{Открывающая сцена}\label{ux43eux442ux43aux440ux44bux432ux430ux44eux449ux430ux44f-ux441ux446ux435ux43dux430-4}

\paragraph{Синопсис сцены}\label{ux441ux438ux43dux43eux43fux441ux438ux441-ux441ux446ux435ux43dux44b-4}

\paragraph{Описание и интерпретация художественных текстов сцены (визуального, шумового и музыкального)}\label{ux43eux43fux438ux441ux430ux43dux438ux435-ux438-ux438ux43dux442ux435ux440ux43fux440ux435ux442ux430ux446ux438ux44f-ux445ux443ux434ux43eux436ux435ux441ux442ux432ux435ux43dux43dux44bux445-ux442ux435ux43aux441ux442ux43eux432-ux441ux446ux435ux43dux44b-ux432ux438ux437ux443ux430ux43bux44cux43dux43eux433ux43e-ux448ux443ux43cux43eux432ux43eux433ux43e-ux438-ux43cux443ux437ux44bux43aux430ux43bux44cux43dux43eux433ux43e-3}

\paragraph{Интерпретация роли музыки сцены в общей композиции фильма}\label{ux438ux43dux442ux435ux440ux43fux440ux435ux442ux430ux446ux438ux44f-ux440ux43eux43bux438-ux43cux443ux437ux44bux43aux438-ux441ux446ux435ux43dux44b-ux432-ux43eux431ux449ux435ux439-ux43aux43eux43cux43fux43eux437ux438ux446ux438ux438-ux444ux438ux43bux44cux43cux430-3}

\subsection{«Клиент всегда мёртв» (2001--2005) \{\#sec.SmEt.SFU\})}\label{ux43aux43bux438ux435ux43dux442-ux432ux441ux435ux433ux434ux430-ux43cux451ux440ux442ux432-20012005-sec.smet.sfu}

\subsubsection{Общие замечания о фильме}\label{ux43eux431ux449ux438ux435-ux437ux430ux43cux435ux447ux430ux43dux438ux44f-ux43e-ux444ux438ux43bux44cux43cux435-4}

Сериал «Клиент всегда мёртв» (\emph{Six Feet Under}) создан Аланом Боллом, американским сценаристом, режиссёром и продюсером, и показан по кабельному телевидению HBO (Home Box Office)\footnote{HBO -- североамериканское кабельное телевидение, заявляющее себя как специализирующееся на драматических сериалах для публики с более тонкими эстетическими и интеллектуальными запросами, чем в мейнстриме телевидения.}.
В сериале 5 сезонов, 63 эпизода и завершённая сюжетная линия, с однозначным эпилогом в последнем эпизоде последнего сезона.
Сериал добился чрезвычайно высоких оценок у публики и критики, стал 36-м в списке «пятидесяти лучших шоу всего времени» согласно журналу \emph{Empire}\footnote{https://www.empireonline.com/movies/features/best-tv-shows-ever/?tv=27/}, был удостоен девяти премий «Эмми», трёх «Золотых глобусов» и других наград.

Создатель сериала, Алан Болл, является и сценаристом «Красоты по-Американски», уже исследовавшей экзистенциальные явления, врывающиеся в повседневную жизнь обычных жителей тихого американского пригорода.
В сериале, «рычагом», сдвигающим жизнь персонажей в экзистенциальную глубину, становится их работа: главные герои являются хозяевами и сотрудниками похоронного бюро «Фишер и сыновья».
Они также являются членами одной семьи Фишеров, что несмотря на сложность их взаимоотношений, создаёт особую психологическую атмосферу интимности и, опять же, открытости, экзистенциальной проницаемости (персонажи знают друг друга с детства и знают «насквозь») в сериале, которую невозможно воспроизвести в историях, где герои -- коллеги или приятели.
Каждый эпизод начинается со смерти человека, «клиента» бюро на ближайшие дни.
Вокруг событий, связанных с хозяйственными вопросами бюро и с общением с родственниками или друзьями покойного, «закручивается» сюжет эпизода.
Основное содержание, даже в наиболее драматически насыщенных сериях, оказывается лирическим или психологическим, касающимся внутренних переживаний персонажей, имеющих регулярно дело с вопросами смерти и вынужденных через эту призму смотреть на свою жизнь.

Как и в «Красоте по-Американски», основная форма подачи драматического материала -- чувственная.
Сюрреалистического эффекта\footnote{Гроссманн в анализе аналогичного приёма у Достоевского, называет это «гротеском» \autocite{grossman.dostoevskiy.1925}.} добивается драма, «выбивая» персонажей из обыденности и ставя их в ситуации экзистенциальные.
Как и в кадрах фильма, над которыми работали Сэм Мендес и блестящий оператор Конрад Холл, в кадрах сериала видно стремление каждый выстроить как художественную картину, часто с сюрреалистическим эффектом как на картинах Рене Магритта.
Особенно насыщена визуальными образами, в т.ч. «соединения несоединимого» заставка сериала с вступительными титрами.

Заставка серий также стала своеобразной революцией в своём жанре, продемонстрировав возможность быть высокохудожественным, содержательным и даже самостоятельным произведением.
Здесь же звучит основная музыкальная тема сериала, написанная Томасом Ньюманом\footnote{Ньюман написал также музыку, использующуюся в эпизодах, но она не включена в официальный саундтрек и не упоминается в титрах. Внутри эпизодов большую роль играют заимствованные эстрадные композиции такие как \emph{«Breathe Me»} Sia Furler, \emph{«Heaven»} группы Lamb, \emph{«Feeling Good»} Нины Симон и др..}, за которую он получил премию «Эмми» и две «Грэмми» за «лучшую инструментальную композицию» и «лучшую инструментальную аранжировку».

Вопреки распространённой практике, Алан Болл попросил композитора, после совместного просмотра пилотного эпизода\footnote{«Пилотным» называется первый эпизод сериала, как правило более длинный и драматически нагруженный, чем последующие.}, сперва написать музыкальную композицию на 90 секунд, чтобы потом наложить на неё изображение.
Композитор вскоре вернулся с 15 набросками, характерных не столько тематически, сколько разнообразием «текстур, звучаний, цвета» \autocite{TN.SFU.2011}, чтобы создатели сериала почувствовали, где выбран правильный «тон» для их произведения.
После обсуждения и отбора подходящего материала, Ньюман написал тему, которую затем передали в \emph{Digital Kitchen} в Сиэтле, где под руководством Дэнни Йонта\footnote{Danny Yount -- ведущий дизайнер в проекте вступительных титров к «Клиент всегда мёртв».}.

Для анализа это особенно интересно, так как даёт возможность увидеть не только как Томас Ньюман интерпретирует изображение, но как его музыка интерпретируется слушателями и соавторами в драматическом и визуальном ключе.
Здесь визуальный ряд целиком, от концепции до итогового вариант, создавался под уже завершённую музыку.

\subsubsection{Открывающая сцена: вступительные титры (заставка)}\label{ux43eux442ux43aux440ux44bux432ux430ux44eux449ux430ux44f-ux441ux446ux435ux43dux430-ux432ux441ux442ux443ux43fux438ux442ux435ux43bux44cux43dux44bux435-ux442ux438ux442ux440ux44b-ux437ux430ux441ux442ux430ux432ux43aux430}

\paragraph{Синопсис сцены}\label{ux441ux438ux43dux43eux43fux441ux438ux441-ux441ux446ux435ux43dux44b-5}

\begin{longtable}[]{@{}ll@{}}
\toprule
\begin{minipage}[b]{0.32\columnwidth}\raggedright\strut
кадр\strut
\end{minipage} & \begin{minipage}[b]{0.62\columnwidth}\raggedright\strut
комментарий\strut
\end{minipage}\tabularnewline
\midrule
\endhead
\begin{minipage}[t]{0.48\columnwidth}\raggedright\strut
0:00--0:05
\includegraphics{IMG/SFU_0-00.png}

\textbf{уход в темноту}\strut
\end{minipage} & \begin{minipage}[t]{0.48\columnwidth}\raggedright\strut
Ворон, пролетающий на фоне
голубого неба.
Одинокое дерево на холме.\strut
\end{minipage}\tabularnewline
\begin{minipage}[t]{0.32\columnwidth}\raggedright\strut
\begin{figure}
\centering
\includegraphics{Ly/SmEt/SFUly_0-00.png}
\caption{}
\end{figure}
\strut
\end{minipage} & \begin{minipage}[t]{0.62\columnwidth}\raggedright\strut
Аккорд челесты в высоком регистре в Es-лидийском.
Медленное затухание на фоне модулирующего пэда.\strut
\end{minipage}\tabularnewline
\begin{minipage}[t]{0.32\columnwidth}\raggedright\strut
0:09--0:10
\includegraphics{IMG/SFU_0-09.jpg}\strut
\end{minipage} & \begin{minipage}[t]{0.62\columnwidth}\raggedright\strut
Расстающиеся руки.
Ладони, моющие друг друга.\strut
\end{minipage}\tabularnewline
\begin{minipage}[t]{0.32\columnwidth}\raggedright\strut
\begin{figure}
\centering
\includegraphics{Ly/SmEt/SFUly_0-09.png}
\caption{}
\end{figure}
\strut
\end{minipage} & \begin{minipage}[t]{0.62\columnwidth}\raggedright\strut
К челесте добавляются «наигрыши» пиццикато с
контрапунктом синкопированного баса.\strut
\end{minipage}\tabularnewline
\begin{minipage}[t]{0.32\columnwidth}\raggedright\strut
0:20
\includegraphics{IMG/SFU_0-20.png}\strut
\end{minipage} & \begin{minipage}[t]{0.62\columnwidth}\raggedright\strut
Ступни тела на каталке, с ярлыком
на большом пальце.\strut
\end{minipage}\tabularnewline
\begin{minipage}[t]{0.32\columnwidth}\raggedright\strut
\begin{figure}
\centering
\includegraphics{Ly/SmEt/SFUly_0-20.png}
\caption{}
\end{figure}
\strut
\end{minipage} & \begin{minipage}[t]{0.62\columnwidth}\raggedright\strut
Второй мотив у пиццикато.\strut
\end{minipage}\tabularnewline
\begin{minipage}[t]{0.32\columnwidth}\raggedright\strut
0:27
\includegraphics{IMG/SFU_0-27.png}\strut
\end{minipage} & \begin{minipage}[t]{0.62\columnwidth}\raggedright\strut
Оборачивающийся вокруг оси вид
голубого неба с облаками.
\textbf{уход в свет}\strut
\end{minipage}\tabularnewline
\begin{minipage}[t]{0.32\columnwidth}\raggedright\strut
\begin{figure}
\centering
\includegraphics{Ly/SmEt/SFUly_0-27.png}
\caption{}
\end{figure}
\strut
\end{minipage} & \begin{minipage}[t]{0.62\columnwidth}\raggedright\strut
Кресчендо атмосферного модулирующего пэда (эмбиента).\strut
\end{minipage}\tabularnewline
\begin{minipage}[t]{0.32\columnwidth}\raggedright\strut
0:29--0:39
\includegraphics{IMG/SFU_0-29.jpg}\strut
\end{minipage} & \begin{minipage}[t]{0.62\columnwidth}\raggedright\strut
Колесо движущейся каталки; каталка
отдаляется от человека на контровом свете;
ступни тела на каталке, едущей к свету.\strut
\end{minipage}\tabularnewline
\begin{minipage}[t]{0.32\columnwidth}\raggedright\strut
\begin{figure}
\centering
\includegraphics{Ly/SmEt/SFUly_0-29.png}
\caption{}
\end{figure}
\strut
\end{minipage} & \begin{minipage}[t]{0.62\columnwidth}\raggedright\strut
Элегическая мелодия oboe d'amore на фоне пиццикато.\strut
\end{minipage}\tabularnewline
\begin{minipage}[t]{0.32\columnwidth}\raggedright\strut
0:43
\includegraphics{IMG/SFU_0-43.png}\strut
\end{minipage} & \begin{minipage}[t]{0.62\columnwidth}\raggedright\strut
Стеклянная колба с прозрачной жидкостью,
через неё видно движение причудливо искажённого
преломлением человека в голубом халате.\strut
\end{minipage}\tabularnewline
\begin{minipage}[t]{0.32\columnwidth}\raggedright\strut
\begin{figure}
\centering
\includegraphics{Ly/SmEt/SFUly_0-43.png}
\caption{}
\end{figure}
\strut
\end{minipage} & \begin{minipage}[t]{0.62\columnwidth}\raggedright\strut
Завершение мелодии гобоя.\strut
\end{minipage}\tabularnewline
\begin{minipage}[t]{0.32\columnwidth}\raggedright\strut
0:48
\includegraphics{IMG/SFU_0-48.png}\strut
\end{minipage} & \begin{minipage}[t]{0.62\columnwidth}\raggedright\strut
Быстро, равномерно понижающийся уровень
голубоватой жидкости в некоем медицинском сосуде.\strut
\end{minipage}\tabularnewline
\begin{minipage}[t]{0.32\columnwidth}\raggedright\strut
\begin{figure}
\centering
\includegraphics{Ly/SmEt/SFUly_0-48.png}
\caption{}
\end{figure}
\strut
\end{minipage} & \begin{minipage}[t]{0.62\columnwidth}\raggedright\strut
Причудливая сонорная текстура «фальшивых»
перкуссии, струнных щипковых и электронных тембров,
с отчётливым «металлическим» призвуком.\strut
\end{minipage}\tabularnewline
\begin{minipage}[t]{0.32\columnwidth}\raggedright\strut
0:51--0:53
\includegraphics{IMG/SFU_0-51.jpg}\strut
\end{minipage} & \begin{minipage}[t]{0.62\columnwidth}\raggedright\strut
Повёрнутая от зрителя к голубоватой стене
голова обнажённого женского тела;
Невидимая рука протирает ваткой лоб над открытым
женским глазом, словно нанося макияж.\strut
\end{minipage}\tabularnewline
\begin{minipage}[t]{0.32\columnwidth}\raggedright\strut
\begin{figure}
\centering
\includegraphics{Ly/SmEt/SFUly_0-20.png}
\caption{}
\end{figure}
\strut
\end{minipage} & \begin{minipage}[t]{0.62\columnwidth}\raggedright\strut
Повторение основного пэттерна пиццикато.\strut
\end{minipage}\tabularnewline
\begin{minipage}[t]{0.32\columnwidth}\raggedright\strut
0:55
\includegraphics{IMG/SFU_0-55.png}\strut
\end{minipage} & \begin{minipage}[t]{0.62\columnwidth}\raggedright\strut
Соединённые руки пожилой женщины, лежащие поверх
старинного вида кружевного платья.\strut
\end{minipage}\tabularnewline
\begin{minipage}[t]{0.32\columnwidth}\raggedright\strut
\begin{figure}
\centering
\includegraphics{Ly/SmEt/SFUly_0-55.png}
\caption{}
\end{figure}
\strut
\end{minipage} & \begin{minipage}[t]{0.62\columnwidth}\raggedright\strut
Причудливое сонорное соло бразильской куики\footnotemark{}.\strut
\end{minipage}
\footnotetext{Куика (cuíca) -- бразильский фрикционный барабан. Звук извлекается трением палочки, соединённой с мембраной барабана. В результате получается «скрипящий», иногда «вскрикивающий» звук определённой высоты. Диапазон инструмента позволяет играть даже примитивные мелодические партии. Куика активно используется в бразильской народной музыке и современной эстрадной, стилизованный под танец самбу.}\tabularnewline
\begin{minipage}[t]{0.32\columnwidth}\raggedright\strut
0:57--1:00
\includegraphics{IMG/SFU_0-57.jpg}\strut
\end{minipage} & \begin{minipage}[t]{0.62\columnwidth}\raggedright\strut
Стремительно вянущие цветы на тёмном фоне.\strut
\end{minipage}\tabularnewline
\begin{minipage}[t]{0.32\columnwidth}\raggedright\strut
\begin{figure}
\centering
\includegraphics{Ly/SmEt/SFUly_0-57.png}
\caption{}
\end{figure}
\strut
\end{minipage} & \begin{minipage}[t]{0.62\columnwidth}\raggedright\strut
Аккорды струнных As--Asus4 на педали пэда, кресчендо.\strut
\end{minipage}\tabularnewline
\begin{minipage}[t]{0.48\columnwidth}\raggedright\strut
1:02--1:10

\begin{figure}
\centering
\includegraphics{IMG/SFU_1-02.jpg}
\caption{}
\end{figure}
\strut
\end{minipage} & \begin{minipage}[t]{0.48\columnwidth}\raggedright\strut
Вид изнутри катафалка: открывается дверь; крупный план
рук, вытаскивающих гроб из машины; надгробия;
катафалк на фоне проносящихся облаков.\strut
\end{minipage}\tabularnewline
\begin{minipage}[t]{0.48\columnwidth}\raggedright\strut
\includegraphics{Ly/SmEt/SFUly_0-57.png}
1:11--1:15

\begin{figure}
\centering
\includegraphics{IMG/SFU_1-11.jpg}
\caption{}
\end{figure}
\strut
\end{minipage} & \begin{minipage}[t]{0.48\columnwidth}\raggedright\strut
Аккорды струнных As--Asus4 на педали пэда, кресчендо.
Старые чёрно-белые фотографии; когти ворона на
надгробии; проносимый по кладбищу гроб.\strut
\end{minipage}\tabularnewline
\begin{minipage}[t]{0.32\columnwidth}\raggedright\strut
\begin{figure}
\centering
\includegraphics{Ly/SmEt/SFUly_1-11.png}
\caption{}
\end{figure}
\strut
\end{minipage} & \begin{minipage}[t]{0.62\columnwidth}\raggedright\strut
Повторение мелодии гобоя (2 предложения).\strut
\end{minipage}\tabularnewline
\begin{minipage}[t]{0.32\columnwidth}\raggedright\strut
1:18--1:20
\includegraphics{IMG/SFU_1-18.jpg}\strut
\end{minipage} & \begin{minipage}[t]{0.62\columnwidth}\raggedright\strut
Голова ворона; голубое небо с облаками.\strut
\end{minipage}\tabularnewline
\begin{minipage}[t]{0.32\columnwidth}\raggedright\strut
\begin{figure}
\centering
\includegraphics{Ly/SmEt/SFUly_1-18.png}
\caption{}
\end{figure}
\strut
\end{minipage} & \begin{minipage}[t]{0.62\columnwidth}\raggedright\strut
Повторение сонорной текстуры «фальшивых» тембров.\strut
\end{minipage}\tabularnewline
\begin{minipage}[t]{0.48\columnwidth}\raggedright\strut
1:20

\begin{figure}
\centering
\includegraphics{IMG/SFU_1-20.png}
\caption{}
\end{figure}
\strut
\end{minipage} & \begin{minipage}[t]{0.48\columnwidth}\raggedright\strut
Ворон близким планом, пролетающий на фоне пейзажа
с одиноким деревом на холме.\strut
\end{minipage}\tabularnewline
\begin{minipage}[t]{0.32\columnwidth}\raggedright\strut
\begin{figure}
\centering
\includegraphics{Ly/SmEt/SFUly_1-20.png}
\caption{}
\end{figure}
\strut
\end{minipage} & \begin{minipage}[t]{0.62\columnwidth}\raggedright\strut
Повторение первого мотива пиццикато
на аккордах челесты.\strut
\end{minipage}\tabularnewline
\begin{minipage}[t]{0.48\columnwidth}\raggedright\strut
1:25

\begin{figure}
\centering
\includegraphics{IMG/SFU_1-25.png}
\caption{}
\end{figure}
\strut
\end{minipage} & \begin{minipage}[t]{0.48\columnwidth}\raggedright\strut
Под деревом ритмично появляются одна за другой
линии логотипа сериала.\strut
\end{minipage}\tabularnewline
\begin{minipage}[t]{0.32\columnwidth}\raggedright\strut
\begin{figure}
\centering
\includegraphics{Ly/SmEt/SFUly_1-25.png}
\caption{}
\end{figure}
\strut
\end{minipage} & \begin{minipage}[t]{0.62\columnwidth}\raggedright\strut
Последние ноты мотива, на которых появляются по часовой
стрелке линии логотипа (появление обозначено в нотах).\strut
\end{minipage}\tabularnewline
\begin{minipage}[t]{0.48\columnwidth}\raggedright\strut
1:27

\begin{figure}
\centering
\includegraphics{IMG/SFU_1-27.png}
\caption{}
\end{figure}
\strut
\end{minipage} & \begin{minipage}[t]{0.48\columnwidth}\raggedright\strut
В лого появляется название сериала.
Пейзаж становится чёрно-белым, словно гравюра,
и стремительно «выцветает», кадр «расстворяется»
в белом фоне, остаются только красные буквы названия.\strut
\end{minipage}\tabularnewline
\begin{minipage}[t]{0.32\columnwidth}\raggedright\strut
\begin{figure}
\centering
\includegraphics{Ly/SmEt/SFUly_1-27.png}
\caption{}
\end{figure}
\strut
\end{minipage} & \begin{minipage}[t]{0.62\columnwidth}\raggedright\strut
Басовая нота и педаль пэда, долго затухающие (13 сек.).\strut
\end{minipage}\tabularnewline
\bottomrule
\end{longtable}

\paragraph{Описание и интерпретация художественных текстов сцены (визуального, шумового и музыкального)}\label{ux43eux43fux438ux441ux430ux43dux438ux435-ux438-ux438ux43dux442ux435ux440ux43fux440ux435ux442ux430ux446ux438ux44f-ux445ux443ux434ux43eux436ux435ux441ux442ux432ux435ux43dux43dux44bux445-ux442ux435ux43aux441ux442ux43eux432-ux441ux446ux435ux43dux44b-ux432ux438ux437ux443ux430ux43bux44cux43dux43eux433ux43e-ux448ux443ux43cux43eux432ux43eux433ux43e-ux438-ux43cux443ux437ux44bux43aux430ux43bux44cux43dux43eux433ux43e-4}

Так как в этом случае изображение создавалось по уже написанной музыке, будем двигаться в анализе от элементов композиции Ньюмана, которая уже привлекала исследователей киноведов и музыковедов \autocite[ Сс. 196--198]{akass.SFU.2005}.

Первые звуки трека нарочито привлекают внимание, -- аккорды челесты громкие, с яркой атакой, и долго затухают, что вызывает нарастание напряжения, невольное ожидание: что будет дальше?
Это нарастание поддержано модулирующим пэдом, в котором выделяются то es'`, то g'', словно 3й и 4й обертоны самой низкой ноты es у челесты.
Здесь естественное затухание, которое погружает слушателя в состояние внутреннего созерцания (что неоднократно использует, например, А. Пярт) превращено в неуловимый, но тревожащий внимание возбудитель.
Слушатель побуждается не прислушаться к себе, а внимательно вслушаться в музыку или всмотреться в визуальный ряд.
В дальнейшем Ньюман неоднократно будет использовать для этого тембры EWI или этнических духовых инструментов, шорохи струн дульцимера или цитры и подобные «щекочащие» слух созвучия, действенные, но неидентифицируемые как отдельные звуки слушателем.

Ещё до создания музыки, в ходе обсуждения с создателями сериала, было решено, что в начале титр должно быть некое созвучие, мгновенно вызывающее образ чего-то небесного, неземного\footnote{Ньюман использует слова `celestial', `heavenly' \autocite{TN.SFU.2011}}, для чего он выбрал лидийский лад в верхнем регистре и тембр челесты.
Челеста и регистр придают звуку «холодный», «свежий» характер, изобилие высоких частот -- ассоциацию с дыханием и открытым, ветренным пространством.
Гармония модальна, «нестабильна», а акцент сильной доли, напротив, устойчив, «несдвигаем».
Тембр, акцент, кластерное созвучие g-a-b, регистр вызывают образ вспышки, или словно разбилось нечто хрупкое, за чем следует рефлексия, процветающй гармониками пэд: что теперь с этим делать?
При этом, благодаря ладу, образ светлый, игривый.

Статика самой музыки уже с этих аккордов (и в дальнейшем статичной в пэттерновости, повторяемости) контрастирует с динамикой тембров: экспрессивным ударом челесты, модулирующим, «дышащим» пэдом, а позже синкопированным ритмом пиццикато и баса.
Ритмический рисунок последних придаёт музыке характер театрально-музыкального представления, где экспрессия и динамика жестов синхронизирована с музыкой.

Далее на «танец» пиццикато с басом ложится мелодия oboe d'amore.
Тембр и первоначальные мотивы примешивают к теме средневекового колорита, а аккомпанемент «тянет» в сторону джазовой стилистике.

Движение пиццикато и гобоя трижды прерывается: кресчендо эмбиента, сонорный брейк фальшивых струнно-щипковых, брейк бразильской куики, переходящий в кресчендо аккордов As--Asus4 у струнных.
Во всех случаях эти связки совмещают в себе характер остановки музыки и динамического развития, кресчендо.
Нарастающая динамика вызывает ощущение «перевёрнутого» звука, разивающегося от затухания к атаке.
Возможно, Ньюман использовал здесь перевёрнутые сэмплы пэда и струнных.
Перевёрнутый тембр вызывает ощущение сюрреалистического замедления, замирания, обращения времени, а нарастающая динамика -- близящегося разрешения, -- время вот-вот вновь потечёт в положенную сторону.
Что и происходит, когда связка разрешается в мелодию гобоя и «танец» пиццикато и баса.

Ещё одна важная деталь -- акцентация сильных долей, несмотря на все синкопы.
Удары челесты и ноты баса, в т.ч. последняя нота «ложатся» точно в долю.
Даже характерный затактовый мотив у гобоя вступает в сильную долю:

\begin{figure}
\centering
\includegraphics{./Ly/SmEt/SFUly_0-29_crop.png}
\caption{}
\end{figure}

Дизайнер Йонт выбрал основой для визуального ряда событийный ряд «из жизни владельца похоронного бюро»: приготовление усопшего к прощанию, церемония прощания и погребение, переплетённый с аллегорическими, метафорическими изображениями: рукопожатие превращающееся в жест прощания, ворон, вянущие цветы, одинокое дерево на холме и др. \autocite{yount.SFUTitles.2012}.
С одной стороны, он увидел в характере музыкальной темы увидел «деловой» характер, «как будто кто-то за работой», с другой -- красоту, изящество и аллегорию, то серьёзную, то насмешливую.
На фоне изображений этих двух пластов -- изящный шрифт титров, то подчёркивающий холодную элегантную красоту форм и цвета в кадре, то иронично «приложенный» к изображению: в виде надписи на бутыли с бальзамирующей жидкостью, в виде имён на надгробьях.

Внешне в визуальном тексте с помощью цветокоррекции выделяются сцены холодного, голубо-зеленоватого света, и тёплого, с изобилием коричневых, золотистых оттенков.
Вокруг них группируются образы: одинокое дерево на холме, бледность кожи, холод больничных коридоров и бальзамирующих жидкостей, и теплота солнечного дня на кладбище, зелёной травы, облаков в свете, дорогого дерева и золотистого металла гроба, сепии старинных фото.
Это отражает сочетание в музыке столь же отчётливо очерченных тёплых и холодных тембров и манер исполнения, например несколько отчуждённой челесты и терпкого, насмешливого гобоя.

Неожиданным сонорным, причудливым по тембрам фрагментам, рифмуются гротескные, пугающие и в то же время изящные кадры из кабинета «реставрационного художника»\footnote{Имеется в виду профессия, занимающаяся приготовлением тела усопшего к последнему прощанию.} и с похорон.
Визуально причудливые краски и формы также как и в музыке облечены в аккуратные, без лишних деталей, художественные формы: руки на платье похожи на фрагмент с классической картины, кадры с вороном на фоне пейзажа с одиноким деревом напоминают о живописи Магритта, и т.д..

Создание видео по музыке позволило Йонту расставить и буквальные, денотативные рифмы с музыкой: бас глиссандирует вниз на тонику -- падает уровень жидкости в сосуде, пэд завершает такт на кресчендо -- кадр уходит в свет; в последний раз звучит чёткий ритм пиццикато и басовая нота на сильную долю завершает композицию -- ритмично появляются грани рамки для названия сериала, а на последней ноте кадр становится чёрно-белым и выцветает вместе с угасанием ноты.

В результате, заставка сериала стала, несмотря на обилие разномастных, гротескных образов, очень целостной.
Этому способствуют фактура: танцевальный ритм и форма Ньюмана превращает весь ролик (успешно воспроизведший эти ритм и форму) в своеобразный балет визуальных образов, каждый из которых -- художественный и яркий, запоминающийся «жест».
Этому способствует и объединённость образов в несколько контрастных комплексов: холод/тепло, забавно/страшно, движение/созерцание, простор/теснота, гладкое/шероховатое.
Все эти комплексы ощущаются буквально физически, как пары контрастных \emph{аффектов}.
Таким же образом организована музыка Ньюмана.
На этом примере можно проследить, как подобное сонорно-аффектное восприятие композитора адекватно прочитывается и интерпретируется дизайнером в соответствующие визуальные образы.

В свою очередь сами эти \emph{аффекты} перемешиваются, неожиданно (синкопированно) возникают, исчезают, сменяют друг друга.
Это создаёт эффект когнитивного диссонанса, гротескной и карнавальной эстетики.

\paragraph{Интерпретация роли музыки сцены в общей композиции фильма (сериала)}\label{ux438ux43dux442ux435ux440ux43fux440ux435ux442ux430ux446ux438ux44f-ux440ux43eux43bux438-ux43cux443ux437ux44bux43aux438-ux441ux446ux435ux43dux44b-ux432-ux43eux431ux449ux435ux439-ux43aux43eux43cux43fux43eux437ux438ux446ux438ux438-ux444ux438ux43bux44cux43cux430-ux441ux435ux440ux438ux430ux43bux430}

Когнитивные диссонансы в изображении и музыки отражают диссонансы в сюжете сериала, где постоянно сталкивается и сопоставляется: серьёзное и комичное, повседневность и экзистенциальность, близость и отчуждение, красота и безобразие, страсть и умиротворение, жизнь и смерть.
Заставка подготавливает к этой гротескно-карнавальной эстетики через сочетание элегантного, «сдержанного», «классического» подхода и гротескной иронии (например имя создателя сериала на надгробье).
Язык заставки, как и сериала, выходит за пределы классического \emph{прекрасного}, и примешивает к красоте \emph{возвышенное}, -- через пугающее, возмущающее, требующее внутренней эстетической (а, по Канту, даже этической \autocite{kant.sujdenie.1790}) работы зрителя.

Об этом говорит и дизайнер, Дэнни Йонт, который в согласии с музыкой, увидел необходимость, как он описал «в красоте» \autocite{yount.SFUTitles.2012}, которая пропитывала и историю Алана Болла, и музыку Ньюмана.
Этот мотив уже раньше появлялся в творчестве Болла и, в связи с этим, у Ньюмана, в «Красоте по-Американски» (подробнее в /ref\{sec:AB\}).
Интересно, как этот карнавальный, забавный, несколько беззаботный, а при том фантастический, сюрреалистический, подход напоминает исследование экзистенциальных тем в живописи Магритта или в музыке -- в \emph{L'Illusionista} или \emph{L'Harem} из саундтрека Нино Рота к «Восемь с половиной».
Зритель уже через заставку готовится к сочетанию жанра фантастического и забавного повествования с драматическим и трагическим содержанием.

При этом, уже по заставке можно понять, что жанры не деформируют, не разоблачают друг друга.
Например, в музыке, их сочетание в цельной композиции, не позволяет им перейти в циничное или сюрреалистическое разоблачение всего произведения как одной гротескной шутки.
Здесь, как и в «Зелёной миле» (см. \ref{sec.SmEt.GM}), сохраняется пафос \emph{возвышенного}.
Юмор не разоблачает силу (в данном случае, силу жизни и смерти, силу и значение экзистенциальной ситуации для человека), а подчёркивает её контрастом.
Этого удаётся добиться за счёт драматургии и композиции заставки (а в последствие, эпизодов сериала).
Начинаясь с возвышенных образов, она проходит через ряд контрастных аффектов, но опять завершает на \emph{возвышенном}: образе одинокого дерева на холме, чистого неба, а в музыке -- аккордов челесты.
На миг пиццикато и упругая финальная нота баса, вместе с игриво появляющейся рамкой логотипа, создают последний «шуточный диссонанс», а затем кадр выцветает как старая фотография и уходит на свет, а музыке позволено долго затухать.

Сам факт композиционной целостности уже позволяет \emph{возвышенному} превалировать.
Обычно циничное и сатирическое разоблачение использует метод включения в себя материала, а затем выставления его на посмешище: гротескное искажение, без права материалу «ответить», т.е. высказаться полностью, в композиционной целостности.
Так делает в своих работах Вуди Аллен, а в музыке -- Д. Д. Шостакович, А. Г. Шнитке, в саундтреках -- Б. Херрман.
Здесь же возвышенное оказывается цельным и, потому, преобладающим, а шуточные и озорные элементы только подчёркивают его, что похоже на композиторско-драматургические приёмы Б. Бриттена и Н. Рото.
К подобной же стилистике тяготеет Э. М. Ремарк, создавая сходную ситуацию для персонжей своего романа «Чёрный обелиск», также сталкивающихся повседневно с людьми в экзистенциальных ситуациях из-за работы в похоронном бюро.

\subsection{«Лемони Сникет: 33 несчастья» (2004)}\label{sec.SmEt.JB}

\subsubsection{Общие замечания о фильме и саундтреке}\label{ux43eux431ux449ux438ux435-ux437ux430ux43cux435ux447ux430ux43dux438ux44f-ux43e-ux444ux438ux43bux44cux43cux435-ux438-ux441ux430ux443ux43dux434ux442ux440ux435ux43aux435}

Lemony Snicket's A Series of Unfortunate Events (в русском прокате: «Лемони Сникет: 33 несчастья») -- фильм 2004 года американского режиссёра Брэда Сильберлинга по одноименной серии книг американского писателя Дэниэла Хэндлера.
Фильм был награждён «Оскаром» за лучший грим и трижды номинирован, в том числе за саундтрек Томаса Ньюмана.
Литературный источник и фильм Сильберлинга примечательны экзотическим жанровым сочетанием: детская книга и взрослый, серьёзный по тону, готический и страшный рассказ о заговорах и убийствах.
Фильм был признан как удачная адаптация этого непростого по жанру и стилистике произведения.
В немалой степени этому способствал саундтрек, так как Томас Ньюман к тому времени уже показал себя как мастер необычных жанровых сочетаний, способный тонко уловить и передать смысловые диссонансы фильма.
Карнавальность музыки, гротескные грим и визуальные эффекты, перформативность (в фильме буквально присутствует рассказчик), блестящий актёрский состав\footnote{Джим Кэрри (граф Олаф), Мэрил Стрип (тётя Жозефина), Билли Конноли (д-р Монтгомери Монтгомери), Тимоти Спалл (Банкир По), Джуд Лоу (Лемони Сникет, рассказчик)} воспроизвели в фильме ту атмосферу постмодернистской иронии, которую создавал в произведении и вокруг него Дэниэл Хэндлер\footnote{Хэндлер создал произведение, в котором не только литературный текст становится источником художественного образа, но и внешние события, связанные с книгой. Сам он является одним из героев своих книг, а рассказчик, Лемони Сникет, по словам Хэндлера -- реально существующим лицом, преследуемым графом Олафом за разоблачение его преступлений.}.

Музыка к «Лемони Сникету» -- одна из более поздних работ Ньюмана.
В это время его композиторский стиль уже хорошо узнаваем и многие подражают его использованию модальной гармонии, пост-минималистической фактуре, экспериментам в инструментовке.
В фильме эти типичные для Ньюмана техники находят применение.
В исполнительском составе присутствуют помимо симфонического оркестра уже привычные для оркестровок Ньюмана (и занимающее больше экранного времени чем оркестр) цимбалы, цитра, дульцимер, скрипка Штроха, EWI\footnote{EWI (Electronic Wind Instrument -- электронный духовой инструмент (англ.)) --- появившийся в 21 веке синтезатор со звукоизвлечением и контролем за высотой звука как у деревянных духовых инструментов. EWI позволяет подражать артикуляции и нюансировке, уникальной для духовых инструментов, и в то же время использовать синтетические тембры. Ньюман стал использовать EWI начиная с «Красоты по-Американски», работая с исполнителем Стивеном Тавалоне (Steve Tavaglione).}.

\subsubsection{Сцена с композицией Hurricane Herman (1:03:08--1:05:05)}\label{ux441ux446ux435ux43dux430-ux441-ux43aux43eux43cux43fux43eux437ux438ux446ux438ux435ux439-hurricane-herman-1030810505}

В сцене детям второй раз за фильм угрожает смертельная опасность, и снова стихийного характера.
В первый раз на них нёсся железнодорожный состав, а теперь они повисли над бушующими чёрными водами озера Лакриоза на жалких остатках дома тёти Жозефины.

«Стихийные» кадры (бездна, общий вид руин дома на шатких сваях, положение безвыходности детей) сопровождаются фанфарой `урагана Герман' и лейтмотивом `скверного начала'.
Далее, после секундного замешательства, дети начинают действовать.
«Озарение» вспыхивает у Вайолетт под эвристический лейтмотив.
Она подвязывает волосы и начинает вместе с Клаусом готовить путь к спасению.
Их действие разворачивает музыку в фугато с переменным размером и решительными синкопами струнных marcato.
Вместе с решительным действием детей, выбивающих одну из опор из-под руин дома, фугато доходит до кульминации и прерывается.
Дом опасно кренится то в сторону берега, то в сторону бездны, что иллюстрирует сонорное поле в оркестре из двух пластов: баса и струнных в высоком регистре.
Глиссандирующий бас и аккорды струнных то раздвигаются, то сдвигаются.
Наконец, с последним движением дома к берегу, бас и струнные разрешаются в тонику и проносится фанфара мажорных аккордов, вновь с сочетаниями гармоний третьей степени родства.

\begin{longtable}[]{@{}ll@{}}
\caption{1:03:08--1:05:05}\tabularnewline
\toprule
\begin{minipage}[b]{0.39\columnwidth}\raggedright\strut
кадр\strut
\end{minipage} & \begin{minipage}[b]{0.31\columnwidth}\raggedright\strut
комментарий\strut
\end{minipage}\tabularnewline
\midrule
\endfirsthead
\toprule
\begin{minipage}[b]{0.39\columnwidth}\raggedright\strut
кадр\strut
\end{minipage} & \begin{minipage}[b]{0.31\columnwidth}\raggedright\strut
комментарий\strut
\end{minipage}\tabularnewline
\midrule
\endhead
\begin{minipage}[t]{0.39\columnwidth}\raggedright\strut
\begin{figure}
\centering
\includegraphics{IMG/LS_HH_HH.png}
\caption{}
\end{figure}
\strut
\end{minipage} & \begin{minipage}[t]{0.31\columnwidth}\raggedright\strut
бездна, общий план\strut
\end{minipage}\tabularnewline
\begin{minipage}[t]{0.39\columnwidth}\raggedright\strut
1:03:08\strut
\end{minipage} & \begin{minipage}[t]{0.31\columnwidth}\raggedright\strut
л-м урагана\strut
\end{minipage}\tabularnewline
\begin{minipage}[t]{0.39\columnwidth}\raggedright\strut
\begin{figure}
\centering
\includegraphics{IMG/LS_HH_BB.png}
\caption{}
\end{figure}
\strut
\end{minipage} & \begin{minipage}[t]{0.31\columnwidth}\raggedright\strut
бездна, общий план\strut
\end{minipage}\tabularnewline
\begin{minipage}[t]{0.39\columnwidth}\raggedright\strut
1:03:19\strut
\end{minipage} & \begin{minipage}[t]{0.31\columnwidth}\raggedright\strut
л-м скверного начала\strut
\end{minipage}\tabularnewline
\begin{minipage}[t]{0.39\columnwidth}\raggedright\strut
\begin{figure}
\centering
\includegraphics{IMG/LS_HH_euristic.png}
\caption{}
\end{figure}
\strut
\end{minipage} & \begin{minipage}[t]{0.31\columnwidth}\raggedright\strut
предметы, изобретение\strut
\end{minipage}\tabularnewline
\begin{minipage}[t]{0.39\columnwidth}\raggedright\strut
1:03:38\strut
\end{minipage} & \begin{minipage}[t]{0.31\columnwidth}\raggedright\strut
л-м эвристический\strut
\end{minipage}\tabularnewline
\begin{minipage}[t]{0.39\columnwidth}\raggedright\strut
\begin{figure}
\centering
\includegraphics{IMG/LS_HH_fugato.png}
\caption{}
\end{figure}
\strut
\end{minipage} & \begin{minipage}[t]{0.31\columnwidth}\raggedright\strut
активное действие\strut
\end{minipage}\tabularnewline
\begin{minipage}[t]{0.39\columnwidth}\raggedright\strut
1:04:00\strut
\end{minipage} & \begin{minipage}[t]{0.31\columnwidth}\raggedright\strut
фугато\strut
\end{minipage}\tabularnewline
\begin{minipage}[t]{0.39\columnwidth}\raggedright\strut
\begin{figure}
\centering
\includegraphics{IMG/LS_HH_suspense.png}
\caption{}
\end{figure}
\strut
\end{minipage} & \begin{minipage}[t]{0.31\columnwidth}\raggedright\strut
дом над бездной\strut
\end{minipage}\tabularnewline
\begin{minipage}[t]{0.39\columnwidth}\raggedright\strut
1:04:35\strut
\end{minipage} & \begin{minipage}[t]{0.31\columnwidth}\raggedright\strut
момент саспенса\strut
\end{minipage}\tabularnewline
\begin{minipage}[t]{0.39\columnwidth}\raggedright\strut
\begin{figure}
\centering
\includegraphics{IMG/LS_HH_Abyss.png}
\caption{}
\end{figure}
\strut
\end{minipage} & \begin{minipage}[t]{0.31\columnwidth}\raggedright\strut
бездна, общий план\strut
\end{minipage}\tabularnewline
\begin{minipage}[t]{0.39\columnwidth}\raggedright\strut
1:04:45\strut
\end{minipage} & \begin{minipage}[t]{0.31\columnwidth}\raggedright\strut
аккорды D--As\strut
\end{minipage}\tabularnewline
\bottomrule
\end{longtable}

В музыке к сцене замечательно сочетание иллюстративности, звукоизобразительности с музыкальностью.
Это особенно заметно в том, что она целиком, в том же виде и хронометраже, что в фильме, вошла в саундтрек как композиция Hurricane Herman.
С одной стороны, композиция полностью соответствует ритму сцены, подхватывая коннотативно лейт-мотивами актанты стихии и персонажей, а денотативно -- иллюстрируя решительные движения детей, медленные, тяжеловесные движения дома, наконец рушащиеся в бездну обломки: тяжёлые сваи -- глиссандирующим басом и тремолло ударных, мелкий мусор -- дрожанием струны дульцимера.
С другой стороны, композиция не просто использует ряд тематических и звукоизобразительных блоков, но развивается внутри себя и имеет собственную форму, которая и позволяет ей существовать автономно в изданном саундтреке.

Эта композиция оказалась настолько удачной, что она же, с небольшими изменениями, была использована в предыдущем «неприятном столкновении» Бодлеров с поездом.
Хотя для сцены с поездом была написана отдельная композиция An Unpleasant Incident Involving a Train, вошедшая в саундтрек, в фильме её заменила Hurricane Herman.

\subsubsection{Сцена кульминационного противостояния с антагонистом: Woeful Wedding}\label{ux441ux446ux435ux43dux430-ux43aux443ux43bux44cux43cux438ux43dux430ux446ux438ux43eux43dux43dux43eux433ux43e-ux43fux440ux43eux442ux438ux432ux43eux441ux442ux43eux44fux43dux438ux44f-ux441-ux430ux43dux442ux430ux433ux43eux43dux438ux441ux442ux43eux43c-woeful-wedding}

Кульминационный эпизод, в которой происходит последнее столкновение протагонистов и антагониста -- представление графа `Удивительная свадьба' («Marvelous Marriage»).
Граф подстроил участие в представлении реального судьи, Джастис Страусс, что даёт его свадьбе с Вайолет в ходе представления реальную юридическую силу, и он становится наследником богатства Бодлеров.
В качестве мер предосторожности, граф взял в заложники маленькую Санни, которую подвесил в птичьей клетке над пропастью.
В этом эпизоде кульминирует и противостояние мира детей и мира взрослых.
Дети потеряли всех своих опекунов-защитников, и даже судья, банкир По, детектив доверяют графу и с удовольствием приходят на представление.
Бодлеры же остаются совершенно одни и единственные скорбят и страшатся развязки.
Пока идёт представление, Клаус один на пустыре за театром пытается спасти ситуацию.

бибэнтри\bibentry{bart.narrativ.1987}

сайтет\citet{bart.mifologii.1957}

сайтеп\citep{bart.mifologii.1957}
сайтгод\citeyear{bart.smert.1968}
сайтавтор\citeauthor{bart.tretiy.2015}
автосайт\autocite{mfiles.TN.2012}

Это столкновение двух миров ярко выражено в визуальном ряду, противопоставляющим тёплые кадры театрального представления с манерными костюмами, гримом и игрой актёров холодным цветам пустыря за домом.
В музыке этим краскам соответствуют бойкий оркестрик из четверых исполнителей (скрипка Штроха, аккордеон, туба, литавры), нанятый Олафом для аккомпанемента пьесе, и «холодное» звучание Woeful Wedding, закадровой музыки, аккомпанирующей действиям Клауса.
Woeful Wedding отображает подлинную сущность сцены, как её видят дети.
Её «холодное» звучание достигнуто использованием определённых тембров и приёмов звукообработки:





\begin{itemize}
\tightlist
\item
  глиссандирующие пэды и струнные в верхнем регистре
\item
  глубокая реверберация, как иней обволакивающая фактуру
\item
  этнические флейты с характерным «воздухом» в звукоизвлечении
\item
  сухие щелчки, шорохи струнных и перкуссии и короткие электронные мелодические звуки
\end{itemize}

Все эти тембры в сочетании с безтерцовыми гармониями вызывают образ одинокий, потерянный в большом и враждебном (холодном) пространстве.
Но, чёткость и острота атак пэттерна дульцимера и его движение между I--IV ступенями придают характер сосредоточенности, мужественной решимости, согласующийся с действиями Клауса на пустыре и попытками Вайолет затянуть пьесу.

С приближением Олафа к победе, созерцательный и несколько заунывный характер Woeful Wedding переходит в буйную вариацию с медными духовыми, ударными и аккордами, заполнившимися минорными терциями.
Когда же Клаус находит доказательства, что граф был виновником пожара и гибели их родителей, и с помощью оптического устройства в башне графа видит пепелище на месте их дома, композиция сперва забирается в самые холодные краски, сосредотачиваясь в верхнем регистре струнных и пэда, и потом разрешается фактурно в печальный, хорал с бесконечно затянутым гармоническим разрешением.

Этот характерный пример показывает, как Ньюман кроме музыкального озвучания сцены, фактически тонирует\footnote{Тонировка -- процесс записи звукового сопровождения фильма, осуществляемый отдельно от съёмки изображения. Тонировка позволяет создавать художественный образ с помощью шумового ряда, выбирая и обрабатывая сэмплы шумов в соответствии с эстетическими задачами.}, денотативно иллюстрирует кадры.
Музыка становится частью звукового дизайна картины, не переставая быть музыкой и сохраняя собственную автономную, развивающуюся форму.
Такой органический сплав, как плод опыта и мастерства композитроа в области киномузыки, придают картине целостность и художественную правдивость.

Теперь рассмотрим весь эпизод, в котором звучит композиция Woeful Wedding, чтобы увидеть в нём музыкальные коннотативные и перформативные знаки.

\begin{longtable}[]{@{}llll@{}}
\caption{1:03:08--1:05:05}\tabularnewline
\toprule
\begin{minipage}[b]{0.15\columnwidth}\raggedright\strut
время\strut
\end{minipage} & \begin{minipage}[b]{0.20\columnwidth}\raggedright\strut
музыка\strut
\end{minipage} & \begin{minipage}[b]{0.27\columnwidth}\raggedright\strut
события в кадре\strut
\end{minipage} & \begin{minipage}[b]{0.27\columnwidth}\raggedright\strut
кадр\strut
\end{minipage}\tabularnewline
\midrule
\endfirsthead
\toprule
\begin{minipage}[b]{0.15\columnwidth}\raggedright\strut
время\strut
\end{minipage} & \begin{minipage}[b]{0.20\columnwidth}\raggedright\strut
музыка\strut
\end{minipage} & \begin{minipage}[b]{0.27\columnwidth}\raggedright\strut
события в кадре\strut
\end{minipage} & \begin{minipage}[b]{0.27\columnwidth}\raggedright\strut
кадр\strut
\end{minipage}\tabularnewline
\midrule
\endhead
\begin{minipage}[t]{0.15\columnwidth}\raggedright\strut
1:15:00\strut
\end{minipage} & \begin{minipage}[t]{0.20\columnwidth}\raggedright\strut
Marvelous Marriage
(диегич.)\strut
\end{minipage} & \begin{minipage}[t]{0.27\columnwidth}\raggedright\strut
афиша, оркестрик графа\strut
\end{minipage} & \begin{minipage}[t]{0.27\columnwidth}\raggedright\strut
\begin{figure}
\centering
\includegraphics{IMG/LS_1:15:00.png}
\caption{}
\end{figure}
\strut
\end{minipage}\tabularnewline
\begin{minipage}[t]{0.15\columnwidth}\raggedright\strut
1:15:08\strut
\end{minipage} & \begin{minipage}[t]{0.20\columnwidth}\raggedright\strut
MM звучит издалека\strut
\end{minipage} & \begin{minipage}[t]{0.27\columnwidth}\raggedright\strut
труппа Олафа переживает
из-за приезда театрального
критика\strut
\end{minipage} & \begin{minipage}[t]{0.27\columnwidth}\raggedright\strut
\begin{figure}
\centering
\includegraphics{IMG/LS_HH_BB.png}
\caption{}
\end{figure}
\strut
\end{minipage}\tabularnewline
\begin{minipage}[t]{0.15\columnwidth}\raggedright\strut
1:15:23\strut
\end{minipage} & \begin{minipage}[t]{0.20\columnwidth}\raggedright\strut
MM совсем далеко\strut
\end{minipage} & \begin{minipage}[t]{0.27\columnwidth}\raggedright\strut
Клаус объясняет сестре
замысел Олафа\strut
\end{minipage} & \begin{minipage}[t]{0.27\columnwidth}\raggedright\strut
\begin{figure}
\centering
\includegraphics{IMG/LS_HH_euristic.png}
\caption{}
\end{figure}
\strut
\end{minipage}\tabularnewline
\begin{minipage}[t]{0.15\columnwidth}\raggedright\strut
1:15:39\strut
\end{minipage} & \begin{minipage}[t]{0.20\columnwidth}\raggedright\strut
л-м Олафа\strut
\end{minipage} & \begin{minipage}[t]{0.27\columnwidth}\raggedright\strut
за кулисами появляется граф,
потом Джастис Страус\strut
\end{minipage} & \begin{minipage}[t]{0.27\columnwidth}\raggedright\strut
\strut
\end{minipage}\tabularnewline
\begin{minipage}[t]{0.15\columnwidth}\raggedright\strut
1:16:40\strut
\end{minipage} & \begin{minipage}[t]{0.20\columnwidth}\raggedright\strut
мажорное разрешение
темы Олафа\strut
\end{minipage} & \begin{minipage}[t]{0.27\columnwidth}\raggedright\strut
граф торжествует: судья
попалась на его обман и
комично переживая идёт
гримироваться\strut
\end{minipage} & \begin{minipage}[t]{0.27\columnwidth}\raggedright\strut
\begin{figure}
\centering
\includegraphics{IMG/LS_HH_fugato.png}
\caption{}
\end{figure}
\strut
\end{minipage}\tabularnewline
\begin{minipage}[t]{0.15\columnwidth}\raggedright\strut
1:16:46\strut
\end{minipage} & \begin{minipage}[t]{0.20\columnwidth}\raggedright\strut
развитие темы графа\strut
\end{minipage} & \begin{minipage}[t]{0.27\columnwidth}\raggedright\strut
граф рассказывает свой план,
упиваясь его гениальностью\strut
\end{minipage} & \begin{minipage}[t]{0.27\columnwidth}\raggedright\strut
\strut
\end{minipage}\tabularnewline
\begin{minipage}[t]{0.15\columnwidth}\raggedright\strut
1:17:21\strut
\end{minipage} & \begin{minipage}[t]{0.20\columnwidth}\raggedright\strut
л-м скверного начала\strut
\end{minipage} & \begin{minipage}[t]{0.27\columnwidth}\raggedright\strut
Олаф угрожает жизни Санни,
если Вайолет не согласится\strut
\end{minipage} & \begin{minipage}[t]{0.27\columnwidth}\raggedright\strut
\strut
\end{minipage}\tabularnewline
\begin{minipage}[t]{0.15\columnwidth}\raggedright\strut
1:17:45\strut
\end{minipage} & \begin{minipage}[t]{0.20\columnwidth}\raggedright\strut
\strut
\end{minipage} & \begin{minipage}[t]{0.27\columnwidth}\raggedright\strut
\strut
\end{minipage} & \begin{minipage}[t]{0.27\columnwidth}\raggedright\strut
\strut
\end{minipage}\tabularnewline
\begin{minipage}[t]{0.15\columnwidth}\raggedright\strut
момент саспенса\strut
\end{minipage} & \begin{minipage}[t]{0.20\columnwidth}\raggedright\strut
момент саспенса\strut
\end{minipage} & \begin{minipage}[t]{0.27\columnwidth}\raggedright\strut
дом над бездной\strut
\end{minipage} & \begin{minipage}[t]{0.27\columnwidth}\raggedright\strut
\begin{figure}
\centering
\includegraphics{IMG/LS_HH_suspense.png}
\caption{}
\end{figure}
\strut
\end{minipage}\tabularnewline
\begin{minipage}[t]{0.15\columnwidth}\raggedright\strut
аккорды D--As\strut
\end{minipage} & \begin{minipage}[t]{0.20\columnwidth}\raggedright\strut
аккорды D--As\strut
\end{minipage} & \begin{minipage}[t]{0.27\columnwidth}\raggedright\strut
бездна, общий план\strut
\end{minipage} & \begin{minipage}[t]{0.27\columnwidth}\raggedright\strut
\begin{figure}
\centering
\includegraphics{IMG/LS_HH_Abyss.png}
\caption{}
\end{figure}
\strut
\end{minipage}\tabularnewline
\bottomrule
\end{longtable}

\paragraph{Синопсис сцены}\label{ux441ux438ux43dux43eux43fux441ux438ux441-ux441ux446ux435ux43dux44b-6}

\paragraph{Описание и интерпретация художественных текстов сцены (визуального, шумового и музыкального)}\label{ux43eux43fux438ux441ux430ux43dux438ux435-ux438-ux438ux43dux442ux435ux440ux43fux440ux435ux442ux430ux446ux438ux44f-ux445ux443ux434ux43eux436ux435ux441ux442ux432ux435ux43dux43dux44bux445-ux442ux435ux43aux441ux442ux43eux432-ux441ux446ux435ux43dux44b-ux432ux438ux437ux443ux430ux43bux44cux43dux43eux433ux43e-ux448ux443ux43cux43eux432ux43eux433ux43e-ux438-ux43cux443ux437ux44bux43aux430ux43bux44cux43dux43eux433ux43e-5}

\paragraph{Интерпретация роли музыки сцены в общей композиции фильма}\label{ux438ux43dux442ux435ux440ux43fux440ux435ux442ux430ux446ux438ux44f-ux440ux43eux43bux438-ux43cux443ux437ux44bux43aux438-ux441ux446ux435ux43dux44b-ux432-ux43eux431ux449ux435ux439-ux43aux43eux43cux43fux43eux437ux438ux446ux438ux438-ux444ux438ux43bux44cux43cux430-4}

\subsection{\texorpdfstring{«007: Координаты ``Скайфол''» (2012) анализ сцены skyfall на Шотландской дороге}{«007: Координаты Скайфол» (2012) анализ сцены skyfall на Шотландской дороге}}\label{sec.SmEt.JB}

\subsubsection{Общие замечания о фильме}\label{ux43eux431ux449ux438ux435-ux437ux430ux43cux435ux447ux430ux43dux438ux44f-ux43e-ux444ux438ux43bux44cux43cux435-5}

Фильм «007: Координаты ``Скайфол''» стал самым кассовым фильмом о Джеймсе Бонде.
Это первый проект в жанре «экшн», с которым пришлось работать Ньюману.
Также, композитор неоднократно упомянал {[}@\ldots{}{]} особую осложнённость работы над фильмом сочетанием высоких ожиданий по качеству и при том узнаваемости «бондовской» музыки.
Характерные музыкальные клише, которые узнают и ассоциируют с «бондианой» большинство зрителей, накладывались на характерные клише жанра боевика, на жёсткие требования кассового мейнстримового фильма, в результате чего композитору как бы и не оставалось места для какого-либо оригинального высказывания.

Однако, в новом фильме «бондианы» у штурвала стоял режиссёр Сэм Мендес, уже работавший с Ньюманом в «Красоте по-Американски», да и «несдвигаемый» образ агента 007 неожиданно начал меняться в 21 веке.
В результате, у Ньюмана появилась возможность для эксперимента с киномузыкой и в таком «каноничном» жанре.
Так как большинство сцен фильма, всё-таки, построены типичным для «бондианы» образом, как зрительно, так и музыкально, рассмотрим контрастную сцену фильма, выделяющуюся и по форме, и по содержанию.

\subsubsection{Контрастная сцена}\label{ux43aux43eux43dux442ux440ux430ux441ux442ux43dux430ux44f-ux441ux446ux435ux43dux430}

\paragraph{Синопсис сцены}\label{ux441ux438ux43dux43eux43fux441ux438ux441-ux441ux446ux435ux43dux44b-7}

\begin{center}
  \begin{longtable}{|p{0.55\textwidth}|p{0.30\textwidth}|}
    \caption{«007: Координаты "Скайфол"»: сцена 1:46:10–1:49:20 }
    \label{tab:longtable}
    \\ \hline
    Кадр & Комментарий  \\
    \hline \endfirsthead
    \subcaption{Продолжение таблицы~\ref{tab:longtable}}
    \\ \hline \endhead
    \hline \subcaption{Продолжение на след. стр.}
    \endfoot
    \hline \endlastfoot

    1:46:22 \newline \parbox[c]{1em}{\includegraphics[width=250pt]{IMG/JB_1-46-22.png}}        
  & «ЭМ» выходит из машины и подходит к БОНДУ. Оба смотрят на туманную шотландскую долину. Бонд вздыхает. \newline
\newline
\begin{center}«ЭМ»\end{center} \newline
Ты вырос в этом месте? \newline
\begin{center}БОНД\end{center} \newline
\begin{center}(утвердительно)\end{center} \newline
Мм.
\begin{center}«ЭМ»\end{center} \newline
Сколько тебе было, когда они умерли? \newline
\begin{center}БОНД\end{center} \newline
\begin{center}(насмешливо)\end{center} \newline
Вы знаете ответ на этот вопрос. Вы знаете всю историю.

 \\

    НОТЫ 
    & Описание музыки. Описание музыки. Описание музыки. \\
    \hline

       1:47:00 \newline \parbox[c]{1em}{\includegraphics[width=250pt]{IMG/JB_1-47-00.png}}        
    & 
\begin{center}«ЭМ»\end{center} \newline
\begin{center}(вздыхает)\end{center} \newline
Из сирот всегда выходят лучшие агенты. \newline
\begin{center}БОНД\end{center} \newline
\begin{center}(коротко улыбается, смотрит в сторону)\end{center} \newline
Приближается буря. \newline
\begin{center}(смотрит на «ЭМ», идёт к машине)\end{center} \newline

\\

    & Со слов «приближается буря» начинается h у струнных, на которое «наплывает» аккорд H-dur у пэда, охватывающего все частоты от низкого гула до высоких «воздушных» отзвуков. На движении машины появляется тягучий лидийский наигрыш oboe d'amore. У пэда медленно противопоставляются гармонии H-dur -- F-dur.  \\
    \hline

1:47:07 \newline \parbox[c]{1em}{\includegraphics[width=250pt]{IMG/JB_1-47-07.png}}
&
Машина с Бондом и «Эм» в полном одиночестве движится долинам и лесным дорогам Шотландии. 

\\
 
& 
\\
    \hline


1:47:43 \newline \parbox[c]{1em}{\includegraphics[width=250pt]{IMG/JB_1-47-43.png}}
&
Машина въезжает через ворота на территорию огромного поместья, щедро покрытую туманом.
Надпись на воротах: Skyfall.
На каменном столбе у ворот – бронзовая статуя благородного оленя.
Вдалеке виднеется большой старинный дом.

\\
НОТЫ
& К аккорду F-dur у пэда на cresc. добавляется медная секция.
\\
    \hline

1:48:12 \newline \parbox[c]{1em}{\includegraphics[width=250pt]{IMG/JB_1-48-12.png}}
&
Остановившись у мрачного особняка, «Эм» и Бонд выходят из машины. \newline
\begin{center}«ЭМ»\end{center} \newline
Боже! Я не удивлена, что ты никогда не возвращался. \newline
\begin{center}БОНД\end{center} \newline
\begin{center}(утвердительно)\end{center} \newline
Мм. \newline

Они медленно входят внутрь. Дом выглядит заброшенным. Бонд настороженно осматривается, «Эм» с любопытством следит за ним.

\\
НОТЫ
& Гобой завершает последние мотивы своего напева и стихает. Противопоставление H-dur -- F-dur разрешается в E-dur с открытием двери дома. Затем вновь напряжение: E-dur -- e-moll, и остаётся только «призрачный» H-ind(4).
\\
\hline

1:49:13 \newline \parbox[c]{1em}{\includegraphics[width=250pt]{IMG/JB_1-49-13.png}}
&
Бонд оборачивается на скрип и видит ружьё, нацеленное на него. «Эм» медленно ведёт руку к пистолету под пальто.
Из темноты выходит старик с суровым лицом, но не угрожающего вида. Это КИНКЕЙД, лесничий Бондов, знавший агента 007 ребёнком.

\begin{center}КИНКЕЙД\end{center} \newline
Джеймс. Джеймс Бонд! \newline
\begin{center}БОНД\end{center} \newline
\begin{center}(поражённый)\end{center} \newline
Бог мой, ты ещё жив? \newline
\begin{center}КИНКЕЙД\end{center} \newline
\begin{center}(опускает ружьё, смеётся)\end{center} \newline
И я тебя рад видеть! \newline
\begin{center}(жмут друг другу руки)\end{center} \newline

\\

& пауза
\\
\hline
  \end{longtable}
\end{center}

\paragraph{Описание и интерпретация художественных текстов сцены (визуального, шумового и музыкального)}\label{ux43eux43fux438ux441ux430ux43dux438ux435-ux438-ux438ux43dux442ux435ux440ux43fux440ux435ux442ux430ux446ux438ux44f-ux445ux443ux434ux43eux436ux435ux441ux442ux432ux435ux43dux43dux44bux445-ux442ux435ux43aux441ux442ux43eux432-ux441ux446ux435ux43dux44b-ux432ux438ux437ux443ux430ux43bux44cux43dux43eux433ux43e-ux448ux443ux43cux43eux432ux43eux433ux43e-ux438-ux43cux443ux437ux44bux43aux430ux43bux44cux43dux43eux433ux43e-6}

Бросаются в глаза рифмы, проходящие через несколько текстов фильма: пасмурное, туманное небо над открытыми ветрам долинами Шотландии соединяется в один образ с гулом «воздушного» пэда и словами протагониста «Приближается буря».
Через слова, эти образы становятся знаками дальнейших коннотаций: финального боя главных героев с преследователями, последнего столкновения с «искушением» агента 007 мыслями о предательстве.
Как характерно в поэтике романтизма, здесь пейзаж оказывается образом внутреннего состояния персонажей, а также внешних событий их жизни:

\begin{itemize}
\tightlist
\item
  прошлого --- пустынный пейзаж «рифмуется» с историей о ранней смерти родителей Бонда
\item
  настоящего --- протагонисты в пустынном месте и физически, и социально -- отрезанные от всякого сообщения с начальством, и душевно -- вынужденные выбираться из лабиринта проблем, в чём им неоткуда ждать помощи
\item
  будущего -- оно столь же туманно и неясно, как дорога, теряющаяся в долине
\end{itemize}

В музыке этому соответствуют:

\begin{itemize}
\tightlist
\item
  тембр oboe d'amore и лидийский лад в мелодии и гармонии
\item
  тембр пэда и противопоставление H-dur -- F-dur
\item
  фактура: яркий сольный инструмент на фоне педали пэда
\item
  акустическое решение: изобилие реверберации
\end{itemize}

Сцена сразу погружает в образность внутреннего мира, совершенно «потустороннего» для «бондианы»: первые кадры с Бондом и «Эм» очень статичны, созерцательны, протекают на фоне паузы в музыке и тишины пейзажа нарушаемой только редким птичьим чириканием.
Когда их общение окончательно приводит зрителя в изумление: открывается что-то из прошлого агента 007, человека без прошлого, да ещё и с мотивом рефлексии, сомнения, -- сцена становится откровенно психологической, построенной на тонких нюансах в выборе слов и в мимике персонажей.
Здесь, с неожиданной улыбкой Бонда и, вместо ответа на слова «Эм» -- фразы о надвигающейся буре, вступает музыка, подхватывая новую образную сферу внутреннего мира протагонистов.

Гобой d'amore, о чьей роли у Ньюмана мы уже говорили в связи с музыкой к сериалу «Клиент всегда мёртв» (см. \ref{sec.SmEt.SFU}), здесь значим и как тембр духового инструмента, в сочетании с короткими ламентозными интонациями фактически подражающий человеческому голосу, и как тембр экзотический, чьё терпкое, «сладковатое» звучание переплетается с лидийским ладом и интонацией увеличенной секунды ges--a, образующийся из-за противопоставления H-dur и F-dur в гаромнии.
Эта терпкость и сладость тембра и лидийского лада, переплетаясь с хроматическим сопоставлением далёких гармоний и тревожным «гулом» и даже дрожанием пэда на низких частотах создаёт гротескный образ сна или потустороннего мира, куда погружаются персонажи.

Образ поддерживается визуально: статика персонажей, глядящих на долину, превращается в статику машины, движущейся по словно бы бесконечной дороге к далёким горам, а потом -- разрешается в кульминацию сцены: въезд в поместье Бондов.
Здесь надпись со звучным названием Skyfall и грозная и гордая бронзовая статуя оленя сопровождаются музыкальной кульминацией: аккорд F-dur у пэда поддержан струнной и медной секциями, что оттягивает центр модальной гармонии от H-dur к напряжённому F-dur, и в целом подчёркивает «грозное» гармоническое противопоставление далёких тональностей.

Надпись «Skyfall» здесь также играет большую роль: она и раскрывает название фильма, неожиданно для зрителя оказываясь не названием ужасного оружия массового поражения или кличкой антагониста, а названием места из прошлого главного героя.
В то же время, здесь расцветают коннотации названия: «падающие небеса» перекликаются и с крушением «незыблимого» «Ми-6», и с неожиданным вопросом главного героя, на чьей стороне он стоит.

В целом, звучание модальной гармонии и простой фактуры (сольный инструмент на фоне педали) создаёт образ архаический, древний, сооветствующий пейзажам и особенно виду старинного поместья Бондов.
Таким образом, место последнего боя героя оказывается не футуристическим полигоном, как в большинстве фильмов серии, а как бы местом из прошлого, из саг и других героических преданий.

\paragraph{Интерпретация роли музыки сцены в общей композиции фильма}\label{ux438ux43dux442ux435ux440ux43fux440ux435ux442ux430ux446ux438ux44f-ux440ux43eux43bux438-ux43cux443ux437ux44bux43aux438-ux441ux446ux435ux43dux44b-ux432-ux43eux431ux449ux435ux439-ux43aux43eux43cux43fux43eux437ux438ux446ux438ux438-ux444ux438ux43bux44cux43cux430-5}

Эта сцена -- крайне необычная для жанра «бондианы» из-за своего «пейзажного» языка и лирического содержания: истории жизни и внутренних переживаний главного героя, выраженных через пейзаж.
Другие «пейзажные» сцены фильмов о Бонде, в т.ч. и в данной картине, как правило ограничиваются короткой грандиозной сценой яркого сеттинга, сопровождаемого бравурной оркестровой музыкой: ночной Гонконг с высоты птичьего полёта, заброшенный город-остров, где базируется главный антагонист и т.д..

Музыка здесь выражает глубинные процессы, происходящие в культуре, так что даже такой «классический» образ постмодерна как агент 007 начинает изменяться.
Это странно, так как «бондиана» представляет собой, фактически, поп-арт, звучно и красочно соединяющий узнаваемые, легко усвояемые знаки постмодернистической мифологии и особенно -- знаки эффективности, власти, сексуальности, независимости.
Эти знаки, как и вся парадигма поп-арта пронизаны импульсом «здесь и сейчас», в них нет места рефлексии над прошлым или планированием изменений в будущем, -- всему, что требует онтологических оснований, отрицаемых в постмодерне.
Джеймс Бонд стал одним из символов этой эпохи, -- истории его бесконечных подвигов, авантюр и любовных приключений можно воспринимать только в рамках циклического, неизменного мира постмодерна, где имеет значение только настоящий миг и имеет силу то, что в нём наиболее эффективно и эффектно.
Фильм «Координаты ``Скайфол''» надламывает эту парадигму, неожиданно добавляя к образу агента 007 прошлое и даже детство (практически недопустимый сюжетный ход, если затем требуется рассказать о беспечных любовных похождениях героя, но вполне подобающий для романтизма), рефлексию над верностью выбранного им пути.
Последнее, фактически, превращает персонажа в нечто совсем новое, так как агент 007 не может сомневаться в своих мотивах, они дают ему безоговорочное основание для его жизни, включая право на любое этическое действие, вплоть до убийства.
Сомнения в этом фундаменте выбивает у персонажа почву из-под ног и вынуждает его рефлексировать уже и над своими действиями.
